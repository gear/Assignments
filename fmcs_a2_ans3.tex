
\begin{coqdoccode}
\coqdocemptyline
\end{coqdoccode}
\subsection*{Q3.1 - Modified Stack Machine.} 

 \noindent  Since we are given new \coqdocdefinition{instrDenote'} function, I am going to change the \coqref{fmcs a2 ans3.ext.compile}{\coqdocdefinition{compile}} and \coqref{fmcs a2 ans3.ext.progDenote}{\coqdocdefinition{progDenote}} function into \coqref{fmcs a2 ans3.compile'}{\coqdocdefinition{compile'}} and \coqref{fmcs a2 ans3.progDenote'}{\coqdocdefinition{progDenote'}} function that accept the new definition of \coqdocdefinition{instrDenote'}. The new functions are defined as follow:\begin{coqdoccode}
\coqdocemptyline
\coqdocnoindent
\coqdockw{Fixpoint} \coqdef{fmcs a2 ans3.progDenote'}{progDenote'}{\coqdocdefinition{progDenote'}} (\coqdocvar{p} : \coqdocdefinition{prog}) (\coqdocvar{s} : \coqdocdefinition{stack}) : \coqexternalref{option}{http://coq.inria.fr/distrib/8.4pl5/stdlib/Coq.Init.Datatypes}{\coqdocinductive{option}} \coqdocdefinition{stack} :=\coqdoceol
\coqdocindent{1.00em}
\coqdockw{match} \coqdocvariable{p} \coqdockw{with}\coqdoceol
\coqdocindent{1.00em}
\ensuremath{|} \coqexternalref{nil}{http://coq.inria.fr/distrib/8.4pl5/stdlib/Coq.Init.Datatypes}{\coqdocconstructor{nil}} \ensuremath{\Rightarrow} \coqexternalref{Some}{http://coq.inria.fr/distrib/8.4pl5/stdlib/Coq.Init.Datatypes}{\coqdocconstructor{Some}} \coqdocvariable{s}\coqdoceol
\coqdocindent{1.00em}
\ensuremath{|} \coqdocvar{i} \coqexternalref{:list scope:x '::' x}{http://coq.inria.fr/distrib/8.4pl5/stdlib/Coq.Init.Datatypes}{\coqdocnotation{::}} \coqdocvar{p'} \ensuremath{\Rightarrow} \coqdockw{match} \coqdocdefinition{instrDenote'} \coqdocvar{i} \coqdocvariable{s} \coqdockw{with}\coqdoceol
\coqdocindent{7.50em}
\ensuremath{|} \coqexternalref{None}{http://coq.inria.fr/distrib/8.4pl5/stdlib/Coq.Init.Datatypes}{\coqdocconstructor{None}} \ensuremath{\Rightarrow} \coqexternalref{None}{http://coq.inria.fr/distrib/8.4pl5/stdlib/Coq.Init.Datatypes}{\coqdocconstructor{None}}\coqdoceol
\coqdocindent{7.50em}
\ensuremath{|} \coqexternalref{Some}{http://coq.inria.fr/distrib/8.4pl5/stdlib/Coq.Init.Datatypes}{\coqdocconstructor{Some}} \coqdocvar{s'} \ensuremath{\Rightarrow} \coqref{fmcs a2 ans3.progDenote'}{\coqdocdefinition{progDenote'}} \coqdocvar{p'} \coqdocvar{s'}\coqdoceol
\coqdocindent{7.50em}
\coqdockw{end}\coqdoceol
\coqdocindent{1.00em}
\coqdockw{end}.\coqdoceol
\coqdocemptyline
\coqdocnoindent
\coqdockw{Fixpoint} \coqdef{fmcs a2 ans3.compile'}{compile'}{\coqdocdefinition{compile'}} (\coqdocvar{e} : \coqdocinductive{exp}) : \coqdocdefinition{prog} :=\coqdoceol
\coqdocindent{1.00em}
\coqdockw{match} \coqdocvariable{e} \coqdockw{with}\coqdoceol
\coqdocindent{1.00em}
\ensuremath{|} \coqdocconstructor{Const} \coqdocvar{n} \ensuremath{\Rightarrow} \coqdocconstructor{iConst} \coqdocvar{n} \coqexternalref{:list scope:x '::' x}{http://coq.inria.fr/distrib/8.4pl5/stdlib/Coq.Init.Datatypes}{\coqdocnotation{::}} \coqexternalref{nil}{http://coq.inria.fr/distrib/8.4pl5/stdlib/Coq.Init.Datatypes}{\coqdocconstructor{nil}}\coqdoceol
\coqdocindent{1.00em}
\ensuremath{|} \coqdocconstructor{Binop} \coqdocvar{b} \coqdocvar{e1} \coqdocvar{e2} \ensuremath{\Rightarrow} \coqexternalref{:list scope:x '++' x}{http://coq.inria.fr/distrib/8.4pl5/stdlib/Coq.Init.Datatypes}{\coqdocnotation{(}}\coqref{fmcs a2 ans3.compile'}{\coqdocdefinition{compile'}} \coqdocvar{e1}\coqexternalref{:list scope:x '++' x}{http://coq.inria.fr/distrib/8.4pl5/stdlib/Coq.Init.Datatypes}{\coqdocnotation{)}} \coqexternalref{:list scope:x '++' x}{http://coq.inria.fr/distrib/8.4pl5/stdlib/Coq.Init.Datatypes}{\coqdocnotation{++}} \coqexternalref{:list scope:x '++' x}{http://coq.inria.fr/distrib/8.4pl5/stdlib/Coq.Init.Datatypes}{\coqdocnotation{(}}\coqref{fmcs a2 ans3.compile'}{\coqdocdefinition{compile'}} \coqdocvar{e2}\coqexternalref{:list scope:x '++' x}{http://coq.inria.fr/distrib/8.4pl5/stdlib/Coq.Init.Datatypes}{\coqdocnotation{)}} \coqexternalref{:list scope:x '++' x}{http://coq.inria.fr/distrib/8.4pl5/stdlib/Coq.Init.Datatypes}{\coqdocnotation{++}} \coqexternalref{:list scope:x '++' x}{http://coq.inria.fr/distrib/8.4pl5/stdlib/Coq.Init.Datatypes}{\coqdocnotation{(}}\coqdocconstructor{iBinop} \coqdocvar{b} \coqexternalref{:list scope:x '::' x}{http://coq.inria.fr/distrib/8.4pl5/stdlib/Coq.Init.Datatypes}{\coqdocnotation{::}} \coqexternalref{nil}{http://coq.inria.fr/distrib/8.4pl5/stdlib/Coq.Init.Datatypes}{\coqdocconstructor{nil}}\coqexternalref{:list scope:x '++' x}{http://coq.inria.fr/distrib/8.4pl5/stdlib/Coq.Init.Datatypes}{\coqdocnotation{)}}\coqdoceol
\coqdocindent{1.00em}
\coqdockw{end}.\coqdoceol
\coqdocemptyline
\end{coqdoccode}
Before going to the proof, I would like to test out the new Stack Machine with few examples of program evaluation and compiler evaluation:\begin{coqdoccode}
\coqdocemptyline
\coqdocnoindent
\coqdockw{Eval} \coqdoctac{simpl} \coqdoctac{in} \coqref{fmcs a2 ans3.progDenote'}{\coqdocdefinition{progDenote'}} (\coqref{fmcs a2 ans3.compile'}{\coqdocdefinition{compile'}} (\coqdocconstructor{Const} 3)) \coqexternalref{nil}{http://coq.inria.fr/distrib/8.4pl5/stdlib/Coq.Init.Datatypes}{\coqdocconstructor{nil}}.\coqdoceol
\end{coqdoccode}
= \coqexternalref{Some}{http://coq.inria.fr/distrib/8.4pl5/stdlib/Coq.Init.Datatypes}{\coqdocconstructor{Some}} (3 :: \coqexternalref{nil}{http://coq.inria.fr/distrib/8.4pl5/stdlib/Coq.Init.Datatypes}{\coqdocconstructor{nil}}) : \coqexternalref{option}{http://coq.inria.fr/distrib/8.4pl5/stdlib/Coq.Init.Datatypes}{\coqdocinductive{option}} \coqdocdefinition{stack} \begin{coqdoccode}
\coqdocemptyline
\coqdocnoindent
\coqdockw{Eval} \coqdoctac{simpl} \coqdoctac{in} \coqref{fmcs a2 ans3.progDenote'}{\coqdocdefinition{progDenote'}} (\coqref{fmcs a2 ans3.compile'}{\coqdocdefinition{compile'}} (\coqdocconstructor{Binop} \coqdocconstructor{Plus} (\coqdocconstructor{Const} 3) (\coqdocconstructor{Const} 4))) \coqexternalref{nil}{http://coq.inria.fr/distrib/8.4pl5/stdlib/Coq.Init.Datatypes}{\coqdocconstructor{nil}}.\coqdoceol
\end{coqdoccode}
= \coqexternalref{Some}{http://coq.inria.fr/distrib/8.4pl5/stdlib/Coq.Init.Datatypes}{\coqdocconstructor{Some}} (7 :: \coqexternalref{nil}{http://coq.inria.fr/distrib/8.4pl5/stdlib/Coq.Init.Datatypes}{\coqdocconstructor{nil}}) : \coqexternalref{option}{http://coq.inria.fr/distrib/8.4pl5/stdlib/Coq.Init.Datatypes}{\coqdocinductive{option}} \coqdocdefinition{stack} \begin{coqdoccode}
\coqdocemptyline
\coqdocnoindent
\coqdockw{Eval} \coqdoctac{simpl} \coqdoctac{in} \coqref{fmcs a2 ans3.progDenote'}{\coqdocdefinition{progDenote'}} (\coqref{fmcs a2 ans3.compile'}{\coqdocdefinition{compile'}} (\coqdocconstructor{Binop} \coqdocconstructor{Times} \coqdoceol
\coqdocindent{6.50em}
(\coqdocconstructor{Binop} \coqdocconstructor{Plus} (\coqdocconstructor{Const} 3) (\coqdocconstructor{Const} 4)) \coqdoceol
\coqdocindent{6.50em}
(\coqdocconstructor{Binop} \coqdocconstructor{Plus} (\coqdocconstructor{Const} 5) (\coqdocconstructor{Const} 6)))) \coqexternalref{nil}{http://coq.inria.fr/distrib/8.4pl5/stdlib/Coq.Init.Datatypes}{\coqdocconstructor{nil}}.\coqdoceol
\end{coqdoccode}
= \coqexternalref{Some}{http://coq.inria.fr/distrib/8.4pl5/stdlib/Coq.Init.Datatypes}{\coqdocconstructor{Some}} (77 :: \coqexternalref{nil}{http://coq.inria.fr/distrib/8.4pl5/stdlib/Coq.Init.Datatypes}{\coqdocconstructor{nil}}) : \coqexternalref{option}{http://coq.inria.fr/distrib/8.4pl5/stdlib/Coq.Init.Datatypes}{\coqdocinductive{option}} \coqdocdefinition{stack} \begin{coqdoccode}
\coqdocemptyline
\coqdocnoindent
\coqdockw{Eval} \coqdoctac{simpl} \coqdoctac{in} \coqref{fmcs a2 ans3.compile'}{\coqdocdefinition{compile'}} (\coqdocconstructor{Binop} \coqdocconstructor{Times} (\coqdocconstructor{Binop} \coqdocconstructor{Plus} (\coqdocconstructor{Const} 2) (\coqdocconstructor{Const} 3)) (\coqdocconstructor{Const} 7)).\coqdoceol
\end{coqdoccode}
= \coqref{fmcs a2 ans3.ext.iConst}{\coqdocconstructor{iConst}} 3 :: \coqref{fmcs a2 ans3.ext.iConst}{\coqdocconstructor{iConst}} 2 :: \coqdocvar{iBinopPlus} :: \coqref{fmcs a2 ans3.ext.iConst}{\coqdocconstructor{iConst}} 7 :: \coqref{fmcs a2 ans3.ext.iBinop}{\coqdocconstructor{iBinop}} \coqref{fmcs a2 ans3.ext.Times}{\coqdocconstructor{Times}} :: \coqexternalref{nil}{http://coq.inria.fr/distrib/8.4pl5/stdlib/Coq.Init.Datatypes}{\coqdocconstructor{nil}} : \coqref{fmcs a2 ans3.ext.prog}{\coqdocdefinition{prog}} 

 Our modified compiler should work with \emph{all} input, therefore we have the compiple'\_correct theorem as follow: \begin{coqdoccode}
\coqdocemptyline
\coqdocnoindent
\coqdockw{Theorem} \coqdef{fmcs a2 ans3.compile' correct}{compile'\_correct}{\coqdoclemma{compile'\_correct}} : \coqdockw{\ensuremath{\forall}} \coqdocvar{e}, \coqref{fmcs a2 ans3.progDenote'}{\coqdocdefinition{progDenote'}} (\coqref{fmcs a2 ans3.compile'}{\coqdocdefinition{compile'}} \coqdocvariable{e}) \coqexternalref{nil}{http://coq.inria.fr/distrib/8.4pl5/stdlib/Coq.Init.Datatypes}{\coqdocconstructor{nil}} \coqexternalref{:type scope:x '=' x}{http://coq.inria.fr/distrib/8.4pl5/stdlib/Coq.Init.Logic}{\coqdocnotation{=}} \coqexternalref{Some}{http://coq.inria.fr/distrib/8.4pl5/stdlib/Coq.Init.Datatypes}{\coqdocconstructor{Some}} (\coqdocdefinition{expDenote} \coqdocvariable{e} \coqexternalref{:list scope:x '::' x}{http://coq.inria.fr/distrib/8.4pl5/stdlib/Coq.Init.Datatypes}{\coqdocnotation{::}} \coqexternalref{nil}{http://coq.inria.fr/distrib/8.4pl5/stdlib/Coq.Init.Datatypes}{\coqdocconstructor{nil}}).\coqdoceol
\coqdocemptyline
\end{coqdoccode}
\noindent To prove this theorem, as in \{cpdt\}, I will use the standard trick of \emph{strengthening the induction hypothesis}. By proving the fact that, given \emph{any} expression, program list state, and stack state, the modified compiler will correctly compile the program to run with \coqref{fmcs a2 ans3.progDenote'}{\coqdocdefinition{progDenote'}}.\begin{coqdoccode}
\coqdocemptyline
\coqdocnoindent
\coqdockw{Lemma} \coqdef{fmcs a2 ans3.compile' correct'}{compile'\_correct'}{\coqdoclemma{compile'\_correct'}} : \coqdockw{\ensuremath{\forall}} \coqdocvar{e} \coqdocvar{p} \coqdocvar{s},\coqdoceol
\coqdocindent{1.00em}
\coqref{fmcs a2 ans3.progDenote'}{\coqdocdefinition{progDenote'}} (\coqref{fmcs a2 ans3.compile'}{\coqdocdefinition{compile'}} \coqdocvariable{e} \coqexternalref{:list scope:x '++' x}{http://coq.inria.fr/distrib/8.4pl5/stdlib/Coq.Init.Datatypes}{\coqdocnotation{++}} \coqdocvariable{p}) \coqdocvariable{s} \coqexternalref{:type scope:x '=' x}{http://coq.inria.fr/distrib/8.4pl5/stdlib/Coq.Init.Logic}{\coqdocnotation{=}} \coqref{fmcs a2 ans3.progDenote'}{\coqdocdefinition{progDenote'}} \coqdocvariable{p} (\coqdocdefinition{expDenote} \coqdocvariable{e} \coqexternalref{:list scope:x '::' x}{http://coq.inria.fr/distrib/8.4pl5/stdlib/Coq.Init.Datatypes}{\coqdocnotation{::}} \coqdocvariable{s}).\coqdoceol
\end{coqdoccode}
\coqdoceol
\coqdocemptyline
\coqdocnoindent
1 \coqdockw{subgoal}\coqdoceol
\coqdocnoindent
\coqdoceol
\coqdocindent{1.00em}
============================\coqdoceol
\coqdocindent{1.50em}
\coqdockw{\ensuremath{\forall}} (\coqdocvariable{e} : \coqref{fmcs a2 ans3.ext.exp}{\coqdocinductive{exp}}) (\coqdocvariable{p} : \coqexternalref{list}{http://coq.inria.fr/distrib/8.4pl5/stdlib/Coq.Init.Datatypes}{\coqdocinductive{list}} \coqref{fmcs a2 ans3.ext.instr}{\coqdocinductive{instr}}) (\coqdocvariable{s} : \coqdocdefinition{stack}),\coqdoceol
\coqdocindent{1.50em}
\coqref{fmcs a2 ans3.progDenote'}{\coqdocdefinition{progDenote'}} (\coqref{fmcs a2 ans3.compile'}{\coqdocdefinition{compile'}} \coqdocvariable{e} ++ \coqdocvariable{p}) \coqdocvariable{s} = \coqref{fmcs a2 ans3.progDenote'}{\coqdocdefinition{progDenote'}} \coqdocvariable{p} (\coqref{fmcs a2 ans3.ext.expDenote}{\coqdocdefinition{expDenote}} \coqdocvariable{e} :: \coqdocvariable{s})  

\coqdocemptyline


 \vspace{0.5em}  \noindent A typical strategy for handling ``\coqdockw{\ensuremath{\forall}}'' is to use \coqdoctac{intros} tactic. However, if we use \coqdoctac{intros} now, before performing \coqdoctac{induction} on expression e, we will have some problem with Coq cannot recognize some pattern later. Therefore, the tactic \coqdoctac{induction} will be used to break down expression e into basic cases first, then I will apply \coqdoctac{intros} tactic for each case.  \vspace{0.5em} \begin{coqdoccode}
\coqdocemptyline
\coqdocnoindent
\coqdoctac{induction} \coqdocvar{e}.\coqdoceol
\end{coqdoccode}


\coqdoceol
\coqdocemptyline
\coqdocnoindent
2 \coqdockw{subgoals}\coqdoceol
\coqdocindent{1.00em}
\coqdocvar{n} : \coqexternalref{nat}{http://coq.inria.fr/distrib/8.4pl5/stdlib/Coq.Init.Datatypes}{\coqdocinductive{nat}}\coqdoceol
\coqdocindent{1.00em}
============================\coqdoceol
\coqdocindent{1.50em}
\coqdockw{\ensuremath{\forall}} (\coqdocvariable{p} : \coqexternalref{list}{http://coq.inria.fr/distrib/8.4pl5/stdlib/Coq.Init.Datatypes}{\coqdocinductive{list}} \coqref{fmcs a2 ans3.ext.instr}{\coqdocinductive{instr}}) (\coqdocvariable{s} : \coqdocdefinition{stack}),\coqdoceol
\coqdocindent{1.50em}
\coqref{fmcs a2 ans3.progDenote'}{\coqdocdefinition{progDenote'}} (\coqref{fmcs a2 ans3.compile'}{\coqdocdefinition{compile'}} (\coqref{fmcs a2 ans3.ext.Const}{\coqdocconstructor{Const}} \coqdocvar{n}) ++ \coqdocvariable{p}) \coqdocvariable{s} = \coqdoceol
\coqdocindent{1.50em}
\coqref{fmcs a2 ans3.progDenote'}{\coqdocdefinition{progDenote'}} \coqdocvariable{p} (\coqref{fmcs a2 ans3.ext.expDenote}{\coqdocdefinition{expDenote}} (\coqref{fmcs a2 ans3.ext.Const}{\coqdocconstructor{Const}} \coqdocvar{n}) :: \coqdocvariable{s})\coqdoceol
\coqdocnoindent
\coqdockw{subgoal} 2 \coqdocvar{is}:\coqdoceol
\coqdocindent{0.50em}
\coqdockw{\ensuremath{\forall}} (\coqdocvariable{p} : \coqexternalref{list}{http://coq.inria.fr/distrib/8.4pl5/stdlib/Coq.Init.Datatypes}{\coqdocinductive{list}} \coqref{fmcs a2 ans3.ext.instr}{\coqdocinductive{instr}}) (\coqdocvariable{s} : \coqdocdefinition{stack}),\coqdoceol
\coqdocindent{0.50em}
\coqref{fmcs a2 ans3.progDenote'}{\coqdocdefinition{progDenote'}} (\coqref{fmcs a2 ans3.compile'}{\coqdocdefinition{compile'}} (\coqref{fmcs a2 ans3.ext.Binop}{\coqdocconstructor{Binop}} \coqdocvariable{b} \coqdocvar{e1} \coqdocvar{e2}) ++ \coqdocvariable{p}) \coqdocvariable{s} = \coqdoceol
\coqdocindent{0.50em}
\coqref{fmcs a2 ans3.progDenote'}{\coqdocdefinition{progDenote'}} \coqdocvariable{p} (\coqref{fmcs a2 ans3.ext.expDenote}{\coqdocdefinition{expDenote}} (\coqref{fmcs a2 ans3.ext.Binop}{\coqdocconstructor{Binop}} \coqdocvariable{b} \coqdocvar{e1} \coqdocvar{e2}) :: \coqdocvariable{s})

\coqdocemptyline




 Assuming we are given some arbitary stack and program: \vspace{0.5em} \begin{coqdoccode}
\coqdocemptyline
\coqdocnoindent
\coqdoctac{intros}.\coqdoceol
\end{coqdoccode}
\coqdoceol
\coqdocemptyline
\coqdocnoindent
2 \coqdockw{subgoals}\coqdoceol
\coqdocindent{1.00em}
\coqdocvar{n} : \coqexternalref{nat}{http://coq.inria.fr/distrib/8.4pl5/stdlib/Coq.Init.Datatypes}{\coqdocinductive{nat}}\coqdoceol
\coqdocindent{1.00em}
\coqdocvariable{p} : \coqexternalref{list}{http://coq.inria.fr/distrib/8.4pl5/stdlib/Coq.Init.Datatypes}{\coqdocinductive{list}} \coqref{fmcs a2 ans3.ext.instr}{\coqdocinductive{instr}}\coqdoceol
\coqdocindent{1.00em}
\coqdocvariable{s} : \coqdocdefinition{stack}\coqdoceol
\coqdocindent{1.00em}
============================\coqdoceol
\coqdocindent{1.50em}
\coqref{fmcs a2 ans3.progDenote'}{\coqdocdefinition{progDenote'}} (\coqref{fmcs a2 ans3.compile'}{\coqdocdefinition{compile'}} (\coqref{fmcs a2 ans3.ext.Const}{\coqdocconstructor{Const}} \coqdocvar{n}) ++ \coqdocvariable{p}) \coqdocvariable{s} =\coqdoceol
\coqdocindent{1.50em}
\coqref{fmcs a2 ans3.progDenote'}{\coqdocdefinition{progDenote'}} \coqdocvariable{p} (\coqref{fmcs a2 ans3.ext.expDenote}{\coqdocdefinition{expDenote}} (\coqref{fmcs a2 ans3.ext.Const}{\coqdocconstructor{Const}} \coqdocvar{n}) :: \coqdocvariable{s})\coqdoceol
\coqdocnoindent
\coqdockw{subgoal} 2 \coqdocvar{is}:\coqdoceol
\coqdocindent{0.50em}
\coqdockw{\ensuremath{\forall}} (\coqdocvariable{p} : \coqexternalref{list}{http://coq.inria.fr/distrib/8.4pl5/stdlib/Coq.Init.Datatypes}{\coqdocinductive{list}} \coqref{fmcs a2 ans3.ext.instr}{\coqdocinductive{instr}}) (\coqdocvariable{s} : \coqdocdefinition{stack}),\coqdoceol
\coqdocindent{0.50em}
\coqref{fmcs a2 ans3.progDenote'}{\coqdocdefinition{progDenote'}} (\coqref{fmcs a2 ans3.compile'}{\coqdocdefinition{compile'}} (\coqref{fmcs a2 ans3.ext.Binop}{\coqdocconstructor{Binop}} \coqdocvariable{b} \coqdocvar{e1} \coqdocvar{e2}) ++ \coqdocvariable{p}) \coqdocvariable{s} =\coqdoceol
\coqdocindent{0.50em}
\coqref{fmcs a2 ans3.progDenote'}{\coqdocdefinition{progDenote'}} \coqdocvariable{p} (\coqref{fmcs a2 ans3.ext.expDenote}{\coqdocdefinition{expDenote}} (\coqref{fmcs a2 ans3.ext.Binop}{\coqdocconstructor{Binop}} \coqdocvariable{b} \coqdocvar{e1} \coqdocvar{e2}) :: \coqdocvariable{s})

\coqdocemptyline




 \noindent The first subgoal can be proved by simplify the function \coqref{fmcs a2 ans3.compile'}{\coqdocdefinition{compile'}} and \coqref{fmcs a2 ans3.ext.expDenote}{\coqdocdefinition{expDenote}}. The tactic named \coqdoctac{simpl} and \coqdoctac{reflexivity} does exactly what we want.  \vspace{0.5em} \begin{coqdoccode}
\coqdocnoindent
\coqdoctac{simpl}.\coqdoceol
\coqdocemptyline
\end{coqdoccode}
\coqdoceol
\coqdocemptyline
\coqdocnoindent
2 \coqdockw{subgoals}\coqdoceol
\coqdocnoindent
\coqdoceol
\coqdocindent{0.50em}
\coqdocvar{n} : \coqexternalref{nat}{http://coq.inria.fr/distrib/8.4pl5/stdlib/Coq.Init.Datatypes}{\coqdocinductive{nat}}\coqdoceol
\coqdocindent{0.50em}
\coqdocvariable{p} : \coqexternalref{list}{http://coq.inria.fr/distrib/8.4pl5/stdlib/Coq.Init.Datatypes}{\coqdocinductive{list}} \coqref{fmcs a2 ans3.ext.instr}{\coqdocinductive{instr}}\coqdoceol
\coqdocindent{0.50em}
\coqdocvariable{s} : \coqdocdefinition{stack}\coqdoceol
\coqdocindent{0.50em}
============================\coqdoceol
\coqdocindent{1.00em}
\coqref{fmcs a2 ans3.progDenote'}{\coqdocdefinition{progDenote'}} \coqdocvariable{p} (\coqdocvar{n} :: \coqdocvariable{s}) = \coqref{fmcs a2 ans3.progDenote'}{\coqdocdefinition{progDenote'}} \coqdocvariable{p} (\coqdocvar{n} :: \coqdocvariable{s})\coqdoceol
\coqdocnoindent
\coqdoceol
\coqdocnoindent
\coqdockw{subgoal} 2 \coqdocvar{is}\coqdoceol
\coqdocindent{0.50em}
\coqdockw{\ensuremath{\forall}} (\coqdocvariable{p} : \coqexternalref{list}{http://coq.inria.fr/distrib/8.4pl5/stdlib/Coq.Init.Datatypes}{\coqdocinductive{list}} \coqref{fmcs a2 ans3.ext.instr}{\coqdocinductive{instr}}) (\coqdocvariable{s} : \coqdocdefinition{stack}),\coqdoceol
\coqdocindent{0.50em}
\coqref{fmcs a2 ans3.progDenote'}{\coqdocdefinition{progDenote'}} (\coqref{fmcs a2 ans3.compile'}{\coqdocdefinition{compile'}} (\coqref{fmcs a2 ans3.ext.Binop}{\coqdocconstructor{Binop}} \coqdocvariable{b} \coqdocvar{e1} \coqdocvar{e2}) ++ \coqdocvariable{p}) \coqdocvariable{s} =\coqdoceol
\coqdocindent{0.50em}
\coqref{fmcs a2 ans3.progDenote'}{\coqdocdefinition{progDenote'}} \coqdocvariable{p} (\coqref{fmcs a2 ans3.ext.expDenote}{\coqdocdefinition{expDenote}} (\coqref{fmcs a2 ans3.ext.Binop}{\coqdocconstructor{Binop}} \coqdocvariable{b} \coqdocvar{e1} \coqdocvar{e2}) :: \coqdocvariable{s})

\coqdocemptyline




 \noindent By using simple \coqdoctac{reflexivity} tactic, I have proved the first subgoal.  \vspace{0.5em} \begin{coqdoccode}
\coqdocemptyline
\coqdocnoindent
\coqdoctac{reflexivity}.\coqdoceol
\coqdocemptyline
\end{coqdoccode}
\coqdoceol
\coqdocemptyline
\coqdocnoindent
1 \coqdockw{subgoal}\coqdoceol
\coqdocnoindent
\coqdoceol
\coqdocindent{0.50em}
\coqdocvariable{b} : \coqref{fmcs a2 ans3.ext.binop}{\coqdocinductive{binop}}\coqdoceol
\coqdocindent{0.50em}
\coqdocvar{e1} : \coqref{fmcs a2 ans3.ext.exp}{\coqdocinductive{exp}}\coqdoceol
\coqdocindent{0.50em}
\coqdocvar{e2} : \coqref{fmcs a2 ans3.ext.exp}{\coqdocinductive{exp}}\coqdoceol
\coqdocindent{0.50em}
\coqdocvar{IHe1} : \coqref{fmcs a2 ans3.progDenote'}{\coqdocdefinition{progDenote'}} (\coqref{fmcs a2 ans3.compile'}{\coqdocdefinition{compile'}} \coqdocvar{e1} ++ \coqdocvariable{p}) \coqdocvariable{s} = \coqref{fmcs a2 ans3.progDenote'}{\coqdocdefinition{progDenote'}} \coqdocvariable{p} (\coqref{fmcs a2 ans3.ext.expDenote}{\coqdocdefinition{expDenote}} \coqdocvar{e1} :: \coqdocvariable{s})\coqdoceol
\coqdocindent{0.50em}
\coqdocvar{IHe2} : \coqref{fmcs a2 ans3.progDenote'}{\coqdocdefinition{progDenote'}} (\coqref{fmcs a2 ans3.compile'}{\coqdocdefinition{compile'}} \coqdocvar{e2} ++ \coqdocvariable{p}) \coqdocvariable{s} = \coqref{fmcs a2 ans3.progDenote'}{\coqdocdefinition{progDenote'}} \coqdocvariable{p} (\coqref{fmcs a2 ans3.ext.expDenote}{\coqdocdefinition{expDenote}} \coqdocvar{e2} :: \coqdocvariable{s})\coqdoceol
\coqdocindent{0.50em}
============================\coqdoceol
\coqdocindent{1.00em}
\coqdockw{\ensuremath{\forall}} (\coqdocvariable{p} : \coqexternalref{list}{http://coq.inria.fr/distrib/8.4pl5/stdlib/Coq.Init.Datatypes}{\coqdocinductive{list}} \coqref{fmcs a2 ans3.ext.instr}{\coqdocinductive{instr}}) (\coqdocvariable{s} : \coqdocdefinition{stack}),\coqdoceol
\coqdocindent{1.00em}
\coqref{fmcs a2 ans3.progDenote'}{\coqdocdefinition{progDenote'}} (\coqref{fmcs a2 ans3.compile'}{\coqdocdefinition{compile'}} (\coqref{fmcs a2 ans3.ext.Binop}{\coqdocconstructor{Binop}} \coqdocvariable{b} \coqdocvar{e1} \coqdocvar{e2}) ++ \coqdocvariable{p}) \coqdocvariable{s} =\coqdoceol
\coqdocindent{1.00em}
\coqref{fmcs a2 ans3.progDenote'}{\coqdocdefinition{progDenote'}} \coqdocvariable{p} (\coqref{fmcs a2 ans3.ext.expDenote}{\coqdocdefinition{expDenote}} (\coqref{fmcs a2 ans3.ext.Binop}{\coqdocconstructor{Binop}} \coqdocvariable{b} \coqdocvar{e1} \coqdocvar{e2}) :: \coqdocvariable{s})

\coqdocemptyline




 \noindent Here we have \coqdocvar{IHe1} and \coqdocvar{IHe2} as two inductive hypothesis. By making the same assumption to handle with ``\coqdockw{\ensuremath{\forall}}'', we have:  \vspace{0.5em} \begin{coqdoccode}
\coqdocemptyline
\coqdocnoindent
\coqdoctac{intros}.\coqdoceol
\coqdocemptyline
\end{coqdoccode}
\coqdoceol
\coqdocemptyline
\coqdocnoindent
1 \coqdockw{subgoal}\coqdoceol
\coqdocnoindent
\coqdoceol
\coqdocindent{0.50em}
\coqdocvariable{b} : \coqref{fmcs a2 ans3.ext.binop}{\coqdocinductive{binop}}\coqdoceol
\coqdocindent{0.50em}
\coqdocvar{e1} : \coqref{fmcs a2 ans3.ext.exp}{\coqdocinductive{exp}}\coqdoceol
\coqdocindent{0.50em}
\coqdocvar{e2} : \coqref{fmcs a2 ans3.ext.exp}{\coqdocinductive{exp}}\coqdoceol
\coqdocindent{0.50em}
\coqdocvar{IHe1} : \coqref{fmcs a2 ans3.progDenote'}{\coqdocdefinition{progDenote'}} (\coqref{fmcs a2 ans3.compile'}{\coqdocdefinition{compile'}} \coqdocvar{e1} ++ \coqdocvariable{p}) \coqdocvariable{s} = \coqref{fmcs a2 ans3.progDenote'}{\coqdocdefinition{progDenote'}} \coqdocvariable{p} (\coqref{fmcs a2 ans3.ext.expDenote}{\coqdocdefinition{expDenote}} \coqdocvar{e1} :: \coqdocvariable{s})\coqdoceol
\coqdocindent{0.50em}
\coqdocvar{IHe2} : \coqref{fmcs a2 ans3.progDenote'}{\coqdocdefinition{progDenote'}} (\coqref{fmcs a2 ans3.compile'}{\coqdocdefinition{compile'}} \coqdocvar{e2} ++ \coqdocvariable{p}) \coqdocvariable{s} = \coqref{fmcs a2 ans3.progDenote'}{\coqdocdefinition{progDenote'}} \coqdocvariable{p} (\coqref{fmcs a2 ans3.ext.expDenote}{\coqdocdefinition{expDenote}} \coqdocvar{e2} :: \coqdocvariable{s})\coqdoceol
\coqdocindent{0.50em}
\coqdocvariable{p} : \coqexternalref{list}{http://coq.inria.fr/distrib/8.4pl5/stdlib/Coq.Init.Datatypes}{\coqdocinductive{list}} \coqref{fmcs a2 ans3.ext.instr}{\coqdocinductive{instr}}\coqdoceol
\coqdocindent{0.50em}
\coqdocvariable{s} : \coqdocdefinition{stack}\coqdoceol
\coqdocindent{0.50em}
============================\coqdoceol
\coqdocindent{1.00em}
\coqref{fmcs a2 ans3.progDenote'}{\coqdocdefinition{progDenote'}} (\coqref{fmcs a2 ans3.compile'}{\coqdocdefinition{compile'}} (\coqref{fmcs a2 ans3.ext.Binop}{\coqdocconstructor{Binop}} \coqdocvariable{b} \coqdocvar{e1} \coqdocvar{e2}) ++ \coqdocvariable{p}) \coqdocvariable{s} =\coqdoceol
\coqdocindent{1.00em}
\coqref{fmcs a2 ans3.progDenote'}{\coqdocdefinition{progDenote'}} \coqdocvariable{p} (\coqref{fmcs a2 ans3.ext.expDenote}{\coqdocdefinition{expDenote}} (\coqref{fmcs a2 ans3.ext.Binop}{\coqdocconstructor{Binop}} \coqdocvariable{b} \coqdocvar{e1} \coqdocvar{e2}) :: \coqdocvariable{s})

\coqdocemptyline




 \noindent The tactic \coqdoctac{simpl} will evaluate the \coqref{fmcs a2 ans3.compile'}{\coqdocdefinition{compile'}} and \coqref{fmcs a2 ans3.ext.expDenote}{\coqdocdefinition{expDenote}} functions:  \vspace{0.5em} \begin{coqdoccode}
\coqdocemptyline
\coqdocnoindent
\coqdoctac{simpl}.\coqdoceol
\coqdocemptyline
\end{coqdoccode}
\coqdoceol
\coqdocemptyline
\coqdocnoindent
1 \coqdockw{subgoal}\coqdoceol
\coqdocnoindent
\coqdoceol
\coqdocindent{0.50em}
\coqdocvariable{b} : \coqref{fmcs a2 ans3.ext.binop}{\coqdocinductive{binop}}\coqdoceol
\coqdocindent{0.50em}
\coqdocvar{e1} : \coqref{fmcs a2 ans3.ext.exp}{\coqdocinductive{exp}}\coqdoceol
\coqdocindent{0.50em}
\coqdocvar{e2} : \coqref{fmcs a2 ans3.ext.exp}{\coqdocinductive{exp}}\coqdoceol
\coqdocindent{0.50em}
\coqdocvar{IHe1} : \coqref{fmcs a2 ans3.progDenote'}{\coqdocdefinition{progDenote'}} (\coqref{fmcs a2 ans3.compile'}{\coqdocdefinition{compile'}} \coqdocvar{e1} ++ \coqdocvariable{p}) \coqdocvariable{s} = \coqref{fmcs a2 ans3.progDenote'}{\coqdocdefinition{progDenote'}} \coqdocvariable{p} (\coqref{fmcs a2 ans3.ext.expDenote}{\coqdocdefinition{expDenote}} \coqdocvar{e1} :: \coqdocvariable{s})\coqdoceol
\coqdocindent{0.50em}
\coqdocvar{IHe2} : \coqref{fmcs a2 ans3.progDenote'}{\coqdocdefinition{progDenote'}} (\coqref{fmcs a2 ans3.compile'}{\coqdocdefinition{compile'}} \coqdocvar{e2} ++ \coqdocvariable{p}) \coqdocvariable{s} = \coqref{fmcs a2 ans3.progDenote'}{\coqdocdefinition{progDenote'}} \coqdocvariable{p} (\coqref{fmcs a2 ans3.ext.expDenote}{\coqdocdefinition{expDenote}} \coqdocvar{e2} :: \coqdocvariable{s})\coqdoceol
\coqdocindent{0.50em}
\coqdocvariable{p} : \coqexternalref{list}{http://coq.inria.fr/distrib/8.4pl5/stdlib/Coq.Init.Datatypes}{\coqdocinductive{list}} \coqref{fmcs a2 ans3.ext.instr}{\coqdocinductive{instr}}\coqdoceol
\coqdocindent{0.50em}
\coqdocvariable{s} : \coqdocdefinition{stack}\coqdoceol
\coqdocindent{0.50em}
============================\coqdoceol
\coqdocindent{1.00em}
\coqref{fmcs a2 ans3.progDenote'}{\coqdocdefinition{progDenote'}} ((\coqref{fmcs a2 ans3.compile'}{\coqdocdefinition{compile'}} \coqdocvar{e1} ++ \coqref{fmcs a2 ans3.compile'}{\coqdocdefinition{compile'}} \coqdocvar{e2} ++ \coqref{fmcs a2 ans3.ext.iBinop}{\coqdocconstructor{iBinop}} \coqdocvariable{b} :: \coqexternalref{nil}{http://coq.inria.fr/distrib/8.4pl5/stdlib/Coq.Init.Datatypes}{\coqdocconstructor{nil}}) ++ \coqdocvariable{p}) \coqdocvariable{s} =\coqdoceol
\coqdocindent{1.00em}
\coqref{fmcs a2 ans3.progDenote'}{\coqdocdefinition{progDenote'}} \coqdocvariable{p} (\coqref{fmcs a2 ans3.ext.binopDenote}{\coqdocdefinition{binopDenote}} \coqdocvariable{b} (\coqref{fmcs a2 ans3.ext.expDenote}{\coqdocdefinition{expDenote}} \coqdocvar{e1}) (\coqref{fmcs a2 ans3.ext.expDenote}{\coqdocdefinition{expDenote}} \coqdocvar{e2}) :: \coqdocvariable{s})

\coqdocemptyline




 \noindent To make the LHS of our target goal similar to the first inductive hypothesis \coqdocvar{IHe1}, I will apply the reverse association rule for \coqexternalref{list}{http://coq.inria.fr/distrib/8.4pl5/stdlib/Coq.Init.Datatypes}{\coqdocinductive{list}} concatenation.  \vspace{0.5em} \begin{coqdoccode}
\coqdocemptyline
\coqdocnoindent
\coqdockw{Check} \coqexternalref{app assoc reverse}{http://coq.inria.fr/distrib/8.4pl5/stdlib/Coq.Lists.List}{\coqdoclemma{app\_assoc\_reverse}}.\coqdoceol
\coqdocemptyline
\end{coqdoccode}
\coqdoceol
\coqdocemptyline
\coqdocnoindent
\coqexternalref{app assoc reverse}{http://coq.inria.fr/distrib/8.4pl5/stdlib/Coq.Lists.List}{\coqdoclemma{app\_assoc\_reverse}}\coqdoceol
\coqdocindent{2.50em}
: \coqdockw{\ensuremath{\forall}} (\coqdocvar{A} : \coqdockw{Type}) (\coqdocvar{l} \coqdocvar{m} \coqdocvar{n} : \coqexternalref{list}{http://coq.inria.fr/distrib/8.4pl5/stdlib/Coq.Init.Datatypes}{\coqdocinductive{list}} \coqdocvar{A}), (\coqdocvar{l} ++ \coqdocvar{m}) ++ \coqdocvar{n} = \coqdocvar{l} ++ \coqdocvar{m} ++ \coqdocvar{n}

\coqdocemptyline


\begin{coqdoccode}
\coqdocemptyline
\coqdocnoindent
\coqdoctac{rewrite} \coqexternalref{app assoc reverse}{http://coq.inria.fr/distrib/8.4pl5/stdlib/Coq.Lists.List}{\coqdoclemma{app\_assoc\_reverse}}.\coqdoceol
\coqdocemptyline
\end{coqdoccode}
\coqdoceol
\coqdocemptyline
\coqdocnoindent
1 \coqdockw{subgoal}\coqdoceol
\coqdocnoindent
\coqdoceol
\coqdocindent{0.50em}
\coqdocvariable{b} : \coqref{fmcs a2 ans3.ext.binop}{\coqdocinductive{binop}}\coqdoceol
\coqdocindent{0.50em}
\coqdocvar{e1} : \coqref{fmcs a2 ans3.ext.exp}{\coqdocinductive{exp}}\coqdoceol
\coqdocindent{0.50em}
\coqdocvar{e2} : \coqref{fmcs a2 ans3.ext.exp}{\coqdocinductive{exp}}\coqdoceol
\coqdocindent{0.50em}
\coqdocvar{IHe1} : \coqref{fmcs a2 ans3.progDenote'}{\coqdocdefinition{progDenote'}} (\coqref{fmcs a2 ans3.compile'}{\coqdocdefinition{compile'}} \coqdocvar{e1} ++ \coqdocvariable{p}) \coqdocvariable{s} = \coqref{fmcs a2 ans3.progDenote'}{\coqdocdefinition{progDenote'}} \coqdocvariable{p} (\coqref{fmcs a2 ans3.ext.expDenote}{\coqdocdefinition{expDenote}} \coqdocvar{e1} :: \coqdocvariable{s})\coqdoceol
\coqdocindent{0.50em}
\coqdocvar{IHe2} : \coqref{fmcs a2 ans3.progDenote'}{\coqdocdefinition{progDenote'}} (\coqref{fmcs a2 ans3.compile'}{\coqdocdefinition{compile'}} \coqdocvar{e2} ++ \coqdocvariable{p}) \coqdocvariable{s} = \coqref{fmcs a2 ans3.progDenote'}{\coqdocdefinition{progDenote'}} \coqdocvariable{p} (\coqref{fmcs a2 ans3.ext.expDenote}{\coqdocdefinition{expDenote}} \coqdocvar{e2} :: \coqdocvariable{s})\coqdoceol
\coqdocindent{0.50em}
\coqdocvariable{p} : \coqexternalref{list}{http://coq.inria.fr/distrib/8.4pl5/stdlib/Coq.Init.Datatypes}{\coqdocinductive{list}} \coqref{fmcs a2 ans3.ext.instr}{\coqdocinductive{instr}}\coqdoceol
\coqdocindent{0.50em}
\coqdocvariable{s} : \coqdocdefinition{stack}\coqdoceol
\coqdocindent{0.50em}
============================\coqdoceol
\coqdocindent{1.00em}
\coqref{fmcs a2 ans3.progDenote'}{\coqdocdefinition{progDenote'}} (\coqref{fmcs a2 ans3.compile'}{\coqdocdefinition{compile'}} \coqdocvar{e1} ++ (\coqref{fmcs a2 ans3.compile'}{\coqdocdefinition{compile'}} \coqdocvar{e2} ++ \coqref{fmcs a2 ans3.ext.iBinop}{\coqdocconstructor{iBinop}} \coqdocvariable{b} :: \coqexternalref{nil}{http://coq.inria.fr/distrib/8.4pl5/stdlib/Coq.Init.Datatypes}{\coqdocconstructor{nil}}) ++ \coqdocvariable{p}) \coqdocvariable{s} =\coqdoceol
\coqdocindent{1.00em}
\coqref{fmcs a2 ans3.progDenote'}{\coqdocdefinition{progDenote'}} \coqdocvariable{p} (\coqref{fmcs a2 ans3.ext.binopDenote}{\coqdocdefinition{binopDenote}} \coqdocvariable{b} (\coqref{fmcs a2 ans3.ext.expDenote}{\coqdocdefinition{expDenote}} \coqdocvar{e1}) (\coqref{fmcs a2 ans3.ext.expDenote}{\coqdocdefinition{expDenote}} \coqdocvar{e2}) :: \coqdocvariable{s})

\coqdocemptyline




 \noindent Now we can apply the inductive hypotheses to ``push'' \coqdocvar{e1} and \coqdocvar{e2} of the LHS to the LHS stack.  \vspace{0.5em} \begin{coqdoccode}
\coqdocemptyline
\coqdocnoindent
\coqdoctac{rewrite} \coqdocvar{IHe1}.\coqdoceol
\coqdocemptyline
\end{coqdoccode}
\coqdoceol
\coqdocemptyline
\coqdocnoindent
1 \coqdockw{subgoal}\coqdoceol
\coqdocnoindent
\coqdoceol
\coqdocindent{0.50em}
\coqdocvariable{b} : \coqref{fmcs a2 ans3.ext.binop}{\coqdocinductive{binop}}\coqdoceol
\coqdocindent{0.50em}
\coqdocvar{e1} : \coqref{fmcs a2 ans3.ext.exp}{\coqdocinductive{exp}}\coqdoceol
\coqdocindent{0.50em}
\coqdocvar{e2} : \coqref{fmcs a2 ans3.ext.exp}{\coqdocinductive{exp}}\coqdoceol
\coqdocindent{0.50em}
\coqdocvar{IHe1} : \coqref{fmcs a2 ans3.progDenote'}{\coqdocdefinition{progDenote'}} (\coqref{fmcs a2 ans3.compile'}{\coqdocdefinition{compile'}} \coqdocvar{e1} ++ \coqdocvariable{p}) \coqdocvariable{s} = \coqref{fmcs a2 ans3.progDenote'}{\coqdocdefinition{progDenote'}} \coqdocvariable{p} (\coqref{fmcs a2 ans3.ext.expDenote}{\coqdocdefinition{expDenote}} \coqdocvar{e1} :: \coqdocvariable{s})\coqdoceol
\coqdocindent{0.50em}
\coqdocvar{IHe2} : \coqref{fmcs a2 ans3.progDenote'}{\coqdocdefinition{progDenote'}} (\coqref{fmcs a2 ans3.compile'}{\coqdocdefinition{compile'}} \coqdocvar{e2} ++ \coqdocvariable{p}) \coqdocvariable{s} = \coqref{fmcs a2 ans3.progDenote'}{\coqdocdefinition{progDenote'}} \coqdocvariable{p} (\coqref{fmcs a2 ans3.ext.expDenote}{\coqdocdefinition{expDenote}} \coqdocvar{e2} :: \coqdocvariable{s})\coqdoceol
\coqdocindent{0.50em}
\coqdocvariable{p} : \coqexternalref{list}{http://coq.inria.fr/distrib/8.4pl5/stdlib/Coq.Init.Datatypes}{\coqdocinductive{list}} \coqref{fmcs a2 ans3.ext.instr}{\coqdocinductive{instr}}\coqdoceol
\coqdocindent{0.50em}
\coqdocvariable{s} : \coqdocdefinition{stack}\coqdoceol
\coqdocindent{0.50em}
============================\coqdoceol
\coqdocindent{1.00em}
\coqref{fmcs a2 ans3.progDenote'}{\coqdocdefinition{progDenote'}} ((\coqref{fmcs a2 ans3.compile'}{\coqdocdefinition{compile'}} \coqdocvar{e2} ++ \coqref{fmcs a2 ans3.ext.iBinop}{\coqdocconstructor{iBinop}} \coqdocvariable{b} :: \coqexternalref{nil}{http://coq.inria.fr/distrib/8.4pl5/stdlib/Coq.Init.Datatypes}{\coqdocconstructor{nil}}) ++ \coqdocvariable{p}) (\coqref{fmcs a2 ans3.ext.expDenote}{\coqdocdefinition{expDenote}} \coqdocvar{e1} :: \coqdocvariable{s}) =\coqdoceol
\coqdocindent{1.00em}
\coqref{fmcs a2 ans3.progDenote'}{\coqdocdefinition{progDenote'}} \coqdocvariable{p} (\coqref{fmcs a2 ans3.ext.binopDenote}{\coqdocdefinition{binopDenote}} \coqdocvariable{b} (\coqref{fmcs a2 ans3.ext.expDenote}{\coqdocdefinition{expDenote}} \coqdocvar{e1}) (\coqref{fmcs a2 ans3.ext.expDenote}{\coqdocdefinition{expDenote}} \coqdocvar{e2}) :: \coqdocvariable{s})

\coqdocemptyline


\begin{coqdoccode}
\coqdocemptyline
\coqdocnoindent
\coqdoctac{rewrite} \coqexternalref{app assoc reverse}{http://coq.inria.fr/distrib/8.4pl5/stdlib/Coq.Lists.List}{\coqdoclemma{app\_assoc\_reverse}}.\coqdoceol
\coqdocemptyline
\coqdocnoindent
\coqdoctac{rewrite} \coqdocvar{IHe2}.\coqdoceol
\coqdocemptyline
\end{coqdoccode}
\coqdoceol
\coqdocemptyline
\coqdocnoindent
1 \coqdockw{subgoal}\coqdoceol
\coqdocnoindent
\coqdoceol
\coqdocindent{0.50em}
\coqdocvariable{b} : \coqref{fmcs a2 ans3.ext.binop}{\coqdocinductive{binop}}\coqdoceol
\coqdocindent{0.50em}
\coqdocvar{e1} : \coqref{fmcs a2 ans3.ext.exp}{\coqdocinductive{exp}}\coqdoceol
\coqdocindent{0.50em}
\coqdocvar{e2} : \coqref{fmcs a2 ans3.ext.exp}{\coqdocinductive{exp}}\coqdoceol
\coqdocindent{0.50em}
\coqdocvar{IHe1} : \coqref{fmcs a2 ans3.progDenote'}{\coqdocdefinition{progDenote'}} (\coqref{fmcs a2 ans3.compile'}{\coqdocdefinition{compile'}} \coqdocvar{e1} ++ \coqdocvariable{p}) \coqdocvariable{s} = \coqref{fmcs a2 ans3.progDenote'}{\coqdocdefinition{progDenote'}} \coqdocvariable{p} (\coqref{fmcs a2 ans3.ext.expDenote}{\coqdocdefinition{expDenote}} \coqdocvar{e1} :: \coqdocvariable{s})\coqdoceol
\coqdocindent{0.50em}
\coqdocvar{IHe2} : \coqref{fmcs a2 ans3.progDenote'}{\coqdocdefinition{progDenote'}} (\coqref{fmcs a2 ans3.compile'}{\coqdocdefinition{compile'}} \coqdocvar{e2} ++ \coqdocvariable{p}) \coqdocvariable{s} = \coqref{fmcs a2 ans3.progDenote'}{\coqdocdefinition{progDenote'}} \coqdocvariable{p} (\coqref{fmcs a2 ans3.ext.expDenote}{\coqdocdefinition{expDenote}} \coqdocvar{e2} :: \coqdocvariable{s})\coqdoceol
\coqdocindent{0.50em}
\coqdocvariable{p} : \coqexternalref{list}{http://coq.inria.fr/distrib/8.4pl5/stdlib/Coq.Init.Datatypes}{\coqdocinductive{list}} \coqref{fmcs a2 ans3.ext.instr}{\coqdocinductive{instr}}\coqdoceol
\coqdocindent{0.50em}
\coqdocvariable{s} : \coqdocdefinition{stack}\coqdoceol
\coqdocindent{0.50em}
============================\coqdoceol
\coqdocindent{1.00em}
\coqref{fmcs a2 ans3.progDenote'}{\coqdocdefinition{progDenote'}} ((\coqref{fmcs a2 ans3.ext.iBinop}{\coqdocconstructor{iBinop}} \coqdocvariable{b} :: \coqexternalref{nil}{http://coq.inria.fr/distrib/8.4pl5/stdlib/Coq.Init.Datatypes}{\coqdocconstructor{nil}}) ++ \coqdocvariable{p}) (\coqref{fmcs a2 ans3.ext.expDenote}{\coqdocdefinition{expDenote}} \coqdocvar{e2} :: \coqref{fmcs a2 ans3.ext.expDenote}{\coqdocdefinition{expDenote}} \coqdocvar{e1} :: \coqdocvariable{s}) =\coqdoceol
\coqdocindent{1.00em}
\coqref{fmcs a2 ans3.progDenote'}{\coqdocdefinition{progDenote'}} \coqdocvariable{p} (\coqref{fmcs a2 ans3.ext.binopDenote}{\coqdocdefinition{binopDenote}} \coqdocvariable{b} (\coqref{fmcs a2 ans3.ext.expDenote}{\coqdocdefinition{expDenote}} \coqdocvar{e1}) (\coqref{fmcs a2 ans3.ext.expDenote}{\coqdocdefinition{expDenote}} \coqdocvar{e2}) :: \coqdocvariable{s})

\coqdocemptyline




 \noindent At this step, we can use the \coqdoctac{simpl} tactic again since it is trivial to evaluate the LHS's \coqref{fmcs a2 ans3.progDenote'}{\coqdocdefinition{progDenote'}} with \coqref{fmcs a2 ans3.ext.iBinop}{\coqdocconstructor{iBinop}} \coqdocvariable{p} :: \coqexternalref{nil}{http://coq.inria.fr/distrib/8.4pl5/stdlib/Coq.Init.Datatypes}{\coqdocconstructor{nil}}.  \vspace{0.5em} \begin{coqdoccode}
\coqdocemptyline
\coqdocnoindent
\coqdoctac{simpl}.\coqdoceol
\coqdocemptyline
\end{coqdoccode}
\coqdoceol
\coqdocemptyline
\coqdocnoindent
1 \coqdockw{subgoal}\coqdoceol
\coqdocnoindent
\coqdoceol
\coqdocindent{0.50em}
\coqdocvariable{b} : \coqref{fmcs a2 ans3.ext.binop}{\coqdocinductive{binop}}\coqdoceol
\coqdocindent{0.50em}
\coqdocvar{e1} : \coqref{fmcs a2 ans3.ext.exp}{\coqdocinductive{exp}}\coqdoceol
\coqdocindent{0.50em}
\coqdocvar{e2} : \coqref{fmcs a2 ans3.ext.exp}{\coqdocinductive{exp}}\coqdoceol
\coqdocindent{0.50em}
\coqdocvar{IHe1} : \coqref{fmcs a2 ans3.progDenote'}{\coqdocdefinition{progDenote'}} (\coqref{fmcs a2 ans3.compile'}{\coqdocdefinition{compile'}} \coqdocvar{e1} ++ \coqdocvariable{p}) \coqdocvariable{s} = \coqref{fmcs a2 ans3.progDenote'}{\coqdocdefinition{progDenote'}} \coqdocvariable{p} (\coqref{fmcs a2 ans3.ext.expDenote}{\coqdocdefinition{expDenote}} \coqdocvar{e1} :: \coqdocvariable{s})\coqdoceol
\coqdocindent{0.50em}
\coqdocvar{IHe2} : \coqref{fmcs a2 ans3.progDenote'}{\coqdocdefinition{progDenote'}} (\coqref{fmcs a2 ans3.compile'}{\coqdocdefinition{compile'}} \coqdocvar{e2} ++ \coqdocvariable{p}) \coqdocvariable{s} = \coqref{fmcs a2 ans3.progDenote'}{\coqdocdefinition{progDenote'}} \coqdocvariable{p} (\coqref{fmcs a2 ans3.ext.expDenote}{\coqdocdefinition{expDenote}} \coqdocvar{e2} :: \coqdocvariable{s})\coqdoceol
\coqdocindent{0.50em}
\coqdocvariable{p} : \coqexternalref{list}{http://coq.inria.fr/distrib/8.4pl5/stdlib/Coq.Init.Datatypes}{\coqdocinductive{list}} \coqref{fmcs a2 ans3.ext.instr}{\coqdocinductive{instr}}\coqdoceol
\coqdocindent{0.50em}
\coqdocvariable{s} : \coqdocdefinition{stack}\coqdoceol
\coqdocindent{0.50em}
============================\coqdoceol
\coqdocindent{1.00em}
\coqref{fmcs a2 ans3.progDenote'}{\coqdocdefinition{progDenote'}} \coqdocvariable{p} (\coqref{fmcs a2 ans3.ext.binopDenote}{\coqdocdefinition{binopDenote}} \coqdocvariable{b} (\coqref{fmcs a2 ans3.ext.expDenote}{\coqdocdefinition{expDenote}} \coqdocvar{e1}) (\coqref{fmcs a2 ans3.ext.expDenote}{\coqdocdefinition{expDenote}} \coqdocvar{e2}) :: \coqdocvariable{s}) =\coqdoceol
\coqdocindent{1.00em}
\coqref{fmcs a2 ans3.progDenote'}{\coqdocdefinition{progDenote'}} \coqdocvariable{p} (\coqref{fmcs a2 ans3.ext.binopDenote}{\coqdocdefinition{binopDenote}} \coqdocvariable{b} (\coqref{fmcs a2 ans3.ext.expDenote}{\coqdocdefinition{expDenote}} \coqdocvar{e1}) (\coqref{fmcs a2 ans3.ext.expDenote}{\coqdocdefinition{expDenote}} \coqdocvar{e2}) :: \coqdocvariable{s})

\coqdocemptyline




 \noindent I comple the proof of this lemma by \coqdoctac{reflexivity} and save it with \coqdockw{Qed}.  \vspace{0.5em} \begin{coqdoccode}
\coqdocemptyline
\coqdocnoindent
\coqdoctac{reflexivity}.\coqdoceol
\coqdocnoindent
\coqdockw{Qed}.\coqdoceol
\coqdocemptyline
\end{coqdoccode}
\coqdoceol
\coqdocemptyline
\coqdocindent{0.50em}
\coqdocvar{introduction} \coqdocvariable{e}.\coqdoceol
\coqdocnoindent
\coqdoceol
\coqdocindent{0.50em}
\coqdoctac{intros}.\coqdoceol
\coqdocindent{0.50em}
\coqdoctac{simpl}.\coqdoceol
\coqdocindent{0.50em}
\coqdoctac{reflexivity}.\coqdoceol
\coqdocnoindent
\coqdoceol
\coqdocindent{0.50em}
\coqdoctac{intros}.\coqdoceol
\coqdocindent{0.50em}
\coqdoctac{simpl}.\coqdoceol
\coqdocindent{0.50em}
\coqdoctac{rewrite} \coqexternalref{app assoc reverse}{http://coq.inria.fr/distrib/8.4pl5/stdlib/Coq.Lists.List}{\coqdoclemma{app\_assoc\_reverse}}.\coqdoceol
\coqdocindent{0.50em}
\coqdoctac{rewrite} \coqdocvar{IHe1}.\coqdoceol
\coqdocindent{0.50em}
\coqdoctac{rewrite} \coqexternalref{app assoc reverse}{http://coq.inria.fr/distrib/8.4pl5/stdlib/Coq.Lists.List}{\coqdoclemma{app\_assoc\_reverse}}.\coqdoceol
\coqdocindent{0.50em}
\coqdoctac{rewrite} \coqdocvar{IHe2}.\coqdoceol
\coqdocindent{0.50em}
\coqdoctac{simpl}.\coqdoceol
\coqdocindent{0.50em}
\coqdoctac{reflexivity}.\coqdoceol
\coqdocnoindent
\coqdoceol
\coqdocnoindent
\coqref{fmcs a2 ans3.compile' correct'}{\coqdoclemma{compile'\_correct'}} \coqdocvar{is} \coqdocvar{defined}

\coqdocemptyline




 \noindent Now we can go back to prove the main theorem:  \vspace{0.5em} \begin{coqdoccode}
\coqdocemptyline
\coqdocnoindent
\coqdockw{Theorem} \coqdef{fmcs a2 ans3.compile' correct}{compile'\_correct}{\coqdoclemma{compile'\_correct}} : \coqdockw{\ensuremath{\forall}} \coqdocvar{e}, \coqref{fmcs a2 ans3.progDenote'}{\coqdocdefinition{progDenote'}} (\coqref{fmcs a2 ans3.compile'}{\coqdocdefinition{compile'}} \coqdocvariable{e}) \coqexternalref{nil}{http://coq.inria.fr/distrib/8.4pl5/stdlib/Coq.Init.Datatypes}{\coqdocconstructor{nil}} \coqexternalref{:type scope:x '=' x}{http://coq.inria.fr/distrib/8.4pl5/stdlib/Coq.Init.Logic}{\coqdocnotation{=}} \coqexternalref{Some}{http://coq.inria.fr/distrib/8.4pl5/stdlib/Coq.Init.Datatypes}{\coqdocconstructor{Some}} (\coqdocdefinition{expDenote} \coqdocvariable{e} \coqexternalref{:list scope:x '::' x}{http://coq.inria.fr/distrib/8.4pl5/stdlib/Coq.Init.Datatypes}{\coqdocnotation{::}} \coqexternalref{nil}{http://coq.inria.fr/distrib/8.4pl5/stdlib/Coq.Init.Datatypes}{\coqdocconstructor{nil}}).\coqdoceol
\coqdocemptyline
\end{coqdoccode}
\noindent Just like with the lemma \coqref{fmcs a2 ans3.compile' correct'}{\coqdoclemma{compile'\_correct'}}, I will firstly introduce the expression \coqdocvariable{e} and then append \coqexternalref{nil}{http://coq.inria.fr/distrib/8.4pl5/stdlib/Coq.Init.Datatypes}{\coqdocconstructor{nil}} to e so that the LHS has the form of \coqref{fmcs a2 ans3.compile' correct'}{\coqdoclemma{compile'\_correct'}}.  \vspace{0.5em} \begin{coqdoccode}
\coqdocemptyline
\coqdocnoindent
\coqdoctac{intros}.\coqdoceol
\coqdocnoindent
\coqdoctac{rewrite} (\coqexternalref{app nil end}{http://coq.inria.fr/distrib/8.4pl5/stdlib/Coq.Lists.List}{\coqdoclemma{app\_nil\_end}} (\coqref{fmcs a2 ans3.compile'}{\coqdocdefinition{compile'}} \coqdocvar{e})).\coqdoceol
\coqdocemptyline
\end{coqdoccode}
\coqdoceol
\coqdocemptyline
\coqdocnoindent
1 \coqdockw{subgoal}\coqdoceol
\coqdocindent{1.00em}
\coqdocvariable{e} : \coqref{fmcs a2 ans3.ext.exp}{\coqdocinductive{exp}}\coqdoceol
\coqdocindent{1.00em}
============================\coqdoceol
\coqdocindent{1.50em}
\coqref{fmcs a2 ans3.progDenote'}{\coqdocdefinition{progDenote'}} (\coqref{fmcs a2 ans3.compile'}{\coqdocdefinition{compile'}} \coqdocvariable{e} ++ \coqexternalref{nil}{http://coq.inria.fr/distrib/8.4pl5/stdlib/Coq.Init.Datatypes}{\coqdocconstructor{nil}}) \coqexternalref{nil}{http://coq.inria.fr/distrib/8.4pl5/stdlib/Coq.Init.Datatypes}{\coqdocconstructor{nil}} = \coqexternalref{Some}{http://coq.inria.fr/distrib/8.4pl5/stdlib/Coq.Init.Datatypes}{\coqdocconstructor{Some}} (\coqref{fmcs a2 ans3.ext.expDenote}{\coqdocdefinition{expDenote}} \coqdocvariable{e} :: \coqexternalref{nil}{http://coq.inria.fr/distrib/8.4pl5/stdlib/Coq.Init.Datatypes}{\coqdocconstructor{nil}})

\coqdocemptyline




 \noindent The theorem is proved by applying lemma \coqref{fmcs a2 ans3.compile' correct'}{\coqdoclemma{compile'\_correct'}} and \coqdocvar{reflexivility}.  \vspace{0.5em} \begin{coqdoccode}
\coqdocemptyline
\coqdocnoindent
\coqdoctac{rewrite} \coqref{fmcs a2 ans3.compile' correct'}{\coqdoclemma{compile'\_correct'}}.\coqdoceol
\coqdocemptyline
\end{coqdoccode}
\coqdoceol
\coqdocemptyline
\coqdocnoindent
1 \coqdockw{subgoal}\coqdoceol
\coqdocindent{1.00em}
\coqdocvariable{e} : \coqref{fmcs a2 ans3.ext.exp}{\coqdocinductive{exp}}\coqdoceol
\coqdocindent{1.00em}
============================\coqdoceol
\coqdocindent{1.50em}
\coqref{fmcs a2 ans3.progDenote'}{\coqdocdefinition{progDenote'}} \coqexternalref{nil}{http://coq.inria.fr/distrib/8.4pl5/stdlib/Coq.Init.Datatypes}{\coqdocconstructor{nil}} (\coqref{fmcs a2 ans3.ext.expDenote}{\coqdocdefinition{expDenote}} \coqdocvariable{e} :: \coqexternalref{nil}{http://coq.inria.fr/distrib/8.4pl5/stdlib/Coq.Init.Datatypes}{\coqdocconstructor{nil}}) = \coqexternalref{Some}{http://coq.inria.fr/distrib/8.4pl5/stdlib/Coq.Init.Datatypes}{\coqdocconstructor{Some}} (\coqref{fmcs a2 ans3.ext.expDenote}{\coqdocdefinition{expDenote}} \coqdocvariable{e} :: \coqexternalref{nil}{http://coq.inria.fr/distrib/8.4pl5/stdlib/Coq.Init.Datatypes}{\coqdocconstructor{nil}})

\coqdocemptyline


\begin{coqdoccode}
\coqdocemptyline
\coqdocnoindent
\coqdoctac{reflexivity}.\coqdoceol
\coqdocnoindent
\coqdockw{Qed}.\coqdoceol
\coqdocemptyline
\end{coqdoccode}
\coqdoceol
\coqdocemptyline
\coqdocindent{0.50em}
\coqdoctac{intros}.\coqdoceol
\coqdocindent{0.50em}
\coqdoctac{rewrite} (\coqexternalref{app nil end}{http://coq.inria.fr/distrib/8.4pl5/stdlib/Coq.Lists.List}{\coqdoclemma{app\_nil\_end}} (\coqref{fmcs a2 ans3.compile'}{\coqdocdefinition{compile'}} \coqdocvariable{e})).\coqdoceol
\coqdocindent{0.50em}
\coqdoctac{rewrite} \coqref{fmcs a2 ans3.compile' correct'}{\coqdoclemma{compile'\_correct'}}.\coqdoceol
\coqdocindent{0.50em}
\coqdoctac{reflexivity}.\coqdoceol
\coqdocnoindent
\coqref{fmcs a2 ans3.compile' correct}{\coqdoclemma{compile'\_correct}} \coqdocvar{is} \coqdocvar{defined}

\coqdocemptyline


 \subsection*{Q3.2 - Extended Stack Machine.} 

 \noindent  The new Stack Machine is defined in module ext as follow: (I keep the definition of \coqdocdefinition{stack} since it is not necessary to re-define it.\begin{coqdoccode}
\coqdocemptyline
\coqdocnoindent
\coqdockw{Module} \coqdef{fmcs a2 ans3.ext}{ext}{\coqdocmodule{ext}}.\coqdoceol
\coqdocemptyline
\coqdocnoindent
\coqdockw{Require} \coqdockw{Import} \coqexternalref{}{http://coq.inria.fr/distrib/8.4pl5/stdlib/Coq.Lists.List}{\coqdoclibrary{List}}.\coqdoceol
\coqdocemptyline
\coqdocnoindent
\coqdockw{Inductive} \coqdef{fmcs a2 ans3.ext.binop}{binop}{\coqdocinductive{binop}} : \coqdockw{Set} := \coqdef{fmcs a2 ans3.ext.Plus}{Plus}{\coqdocconstructor{Plus}} \ensuremath{|} \coqdef{fmcs a2 ans3.ext.Times}{Times}{\coqdocconstructor{Times}} \ensuremath{|} \coqdef{fmcs a2 ans3.ext.Minus}{Minus}{\coqdocconstructor{Minus}}.\coqdoceol
\coqdocnoindent
\coqdockw{Definition} \coqdef{fmcs a2 ans3.ext.binopDenote}{binopDenote}{\coqdocdefinition{binopDenote}} (\coqdocvar{b}:\coqref{fmcs a2 ans3.ext.binop}{\coqdocinductive{binop}}) : \coqexternalref{nat}{http://coq.inria.fr/distrib/8.4pl5/stdlib/Coq.Init.Datatypes}{\coqdocinductive{nat}} \ensuremath{\rightarrow} \coqexternalref{nat}{http://coq.inria.fr/distrib/8.4pl5/stdlib/Coq.Init.Datatypes}{\coqdocinductive{nat}} \ensuremath{\rightarrow} \coqexternalref{nat}{http://coq.inria.fr/distrib/8.4pl5/stdlib/Coq.Init.Datatypes}{\coqdocinductive{nat}} :=\coqdoceol
\coqdocindent{1.00em}
\coqdockw{match} \coqdocvariable{b} \coqdockw{with}\coqdoceol
\coqdocindent{1.00em}
\ensuremath{|} \coqref{fmcs a2 ans3.ext.Plus}{\coqdocconstructor{Plus}} \ensuremath{\Rightarrow} \coqexternalref{plus}{http://coq.inria.fr/distrib/8.4pl5/stdlib/Coq.Init.Peano}{\coqdocdefinition{plus}}\coqdoceol
\coqdocindent{1.00em}
\ensuremath{|} \coqref{fmcs a2 ans3.ext.Times}{\coqdocconstructor{Times}} \ensuremath{\Rightarrow} \coqexternalref{mult}{http://coq.inria.fr/distrib/8.4pl5/stdlib/Coq.Init.Peano}{\coqdocdefinition{mult}}\coqdoceol
\coqdocindent{1.00em}
\ensuremath{|} \coqref{fmcs a2 ans3.ext.Minus}{\coqdocconstructor{Minus}} \ensuremath{\Rightarrow} \coqexternalref{minus}{http://coq.inria.fr/distrib/8.4pl5/stdlib/Coq.Init.Peano}{\coqdocdefinition{minus}}\coqdoceol
\coqdocindent{1.00em}
\coqdockw{end}.\coqdoceol
\coqdocnoindent
\coqdockw{Inductive} \coqdef{fmcs a2 ans3.ext.exp}{exp}{\coqdocinductive{exp}} : \coqdockw{Set} :=\coqdoceol
\coqdocindent{1.00em}
\ensuremath{|} \coqdef{fmcs a2 ans3.ext.Const}{Const}{\coqdocconstructor{Const}} : \coqexternalref{nat}{http://coq.inria.fr/distrib/8.4pl5/stdlib/Coq.Init.Datatypes}{\coqdocinductive{nat}} \ensuremath{\rightarrow} \coqref{fmcs a2 ans3.exp}{\coqdocinductive{exp}}\coqdoceol
\coqdocindent{1.00em}
\ensuremath{|} \coqdef{fmcs a2 ans3.ext.Binop}{Binop}{\coqdocconstructor{Binop}} : \coqref{fmcs a2 ans3.ext.binop}{\coqdocinductive{binop}} \ensuremath{\rightarrow} \coqref{fmcs a2 ans3.exp}{\coqdocinductive{exp}} \ensuremath{\rightarrow} \coqref{fmcs a2 ans3.exp}{\coqdocinductive{exp}} \ensuremath{\rightarrow} \coqref{fmcs a2 ans3.exp}{\coqdocinductive{exp}}.\coqdoceol
\coqdocnoindent
\coqdockw{Fixpoint} \coqdef{fmcs a2 ans3.ext.expDenote}{expDenote}{\coqdocdefinition{expDenote}} (\coqdocvar{e}:\coqref{fmcs a2 ans3.ext.exp}{\coqdocinductive{exp}}) : \coqexternalref{nat}{http://coq.inria.fr/distrib/8.4pl5/stdlib/Coq.Init.Datatypes}{\coqdocinductive{nat}} :=\coqdoceol
\coqdocindent{1.00em}
\coqdockw{match} \coqdocvariable{e} \coqdockw{with}\coqdoceol
\coqdocindent{1.00em}
\ensuremath{|} \coqref{fmcs a2 ans3.ext.Const}{\coqdocconstructor{Const}} \coqdocvar{n} \ensuremath{\Rightarrow} \coqdocvar{n}\coqdoceol
\coqdocindent{1.00em}
\ensuremath{|} \coqref{fmcs a2 ans3.ext.Binop}{\coqdocconstructor{Binop}} \coqdocvar{b} \coqdocvar{e1} \coqdocvar{e2} \ensuremath{\Rightarrow} (\coqref{fmcs a2 ans3.ext.binopDenote}{\coqdocdefinition{binopDenote}} \coqdocvar{b}) (\coqref{fmcs a2 ans3.expDenote}{\coqdocdefinition{expDenote}} \coqdocvar{e1}) (\coqref{fmcs a2 ans3.expDenote}{\coqdocdefinition{expDenote}} \coqdocvar{e2})\coqdoceol
\coqdocindent{1.00em}
\coqdockw{end}.\coqdoceol
\coqdocnoindent
\coqdockw{Inductive} \coqdef{fmcs a2 ans3.ext.instr}{instr}{\coqdocinductive{instr}} : \coqdockw{Set} :=\coqdoceol
\coqdocindent{1.00em}
\ensuremath{|} \coqdef{fmcs a2 ans3.ext.iConst}{iConst}{\coqdocconstructor{iConst}} : \coqexternalref{nat}{http://coq.inria.fr/distrib/8.4pl5/stdlib/Coq.Init.Datatypes}{\coqdocinductive{nat}} \ensuremath{\rightarrow} \coqref{fmcs a2 ans3.instr}{\coqdocinductive{instr}}\coqdoceol
\coqdocindent{1.00em}
\ensuremath{|} \coqdef{fmcs a2 ans3.ext.iBinop}{iBinop}{\coqdocconstructor{iBinop}} : \coqref{fmcs a2 ans3.ext.binop}{\coqdocinductive{binop}} \ensuremath{\rightarrow} \coqref{fmcs a2 ans3.instr}{\coqdocinductive{instr}}.\coqdoceol
\coqdocnoindent
\coqdockw{Definition} \coqdef{fmcs a2 ans3.ext.prog}{prog}{\coqdocdefinition{prog}} := \coqexternalref{list}{http://coq.inria.fr/distrib/8.4pl5/stdlib/Coq.Init.Datatypes}{\coqdocinductive{list}} \coqref{fmcs a2 ans3.ext.instr}{\coqdocinductive{instr}}.\coqdoceol
\coqdocnoindent
\coqdockw{Definition}  \coqdef{fmcs a2 ans3.ext.instrDenote}{instrDenote}{\coqdocdefinition{instrDenote}} (\coqdocvar{i} : \coqref{fmcs a2 ans3.ext.instr}{\coqdocinductive{instr}}) (\coqdocvar{s} : \coqdocdefinition{stack}) : \coqexternalref{option}{http://coq.inria.fr/distrib/8.4pl5/stdlib/Coq.Init.Datatypes}{\coqdocinductive{option}} \coqdocdefinition{stack} :=\coqdoceol
\coqdocindent{1.00em}
\coqdockw{match} \coqdocvariable{i} \coqdockw{with}\coqdoceol
\coqdocindent{1.00em}
\ensuremath{|} \coqref{fmcs a2 ans3.ext.iConst}{\coqdocconstructor{iConst}} \coqdocvar{n} \ensuremath{\Rightarrow} \coqexternalref{Some}{http://coq.inria.fr/distrib/8.4pl5/stdlib/Coq.Init.Datatypes}{\coqdocconstructor{Some}} (\coqdocvar{n} \coqexternalref{:list scope:x '::' x}{http://coq.inria.fr/distrib/8.4pl5/stdlib/Coq.Init.Datatypes}{\coqdocnotation{::}} \coqdocvariable{s})\coqdoceol
\coqdocindent{1.00em}
\ensuremath{|} \coqref{fmcs a2 ans3.ext.iBinop}{\coqdocconstructor{iBinop}} \coqdocvar{b} \ensuremath{\Rightarrow} \coqdockw{match} \coqdocvariable{s} \coqdockw{with}\coqdoceol
\coqdocindent{8.00em}
\ensuremath{|} \coqdocvar{arg2} \coqexternalref{:list scope:x '::' x}{http://coq.inria.fr/distrib/8.4pl5/stdlib/Coq.Init.Datatypes}{\coqdocnotation{::}} \coqdocvar{arg1} \coqexternalref{:list scope:x '::' x}{http://coq.inria.fr/distrib/8.4pl5/stdlib/Coq.Init.Datatypes}{\coqdocnotation{::}} \coqdocvar{s'} \ensuremath{\Rightarrow} \coqexternalref{Some}{http://coq.inria.fr/distrib/8.4pl5/stdlib/Coq.Init.Datatypes}{\coqdocconstructor{Some}} ((\coqref{fmcs a2 ans3.ext.binopDenote}{\coqdocdefinition{binopDenote}} \coqdocvar{b}) \coqdocvar{arg1} \coqdocvar{arg2} \coqexternalref{:list scope:x '::' x}{http://coq.inria.fr/distrib/8.4pl5/stdlib/Coq.Init.Datatypes}{\coqdocnotation{::}} \coqdocvar{s'})\coqdoceol
\coqdocindent{8.00em}
\ensuremath{|} \coqdocvar{\_} \ensuremath{\Rightarrow} \coqexternalref{None}{http://coq.inria.fr/distrib/8.4pl5/stdlib/Coq.Init.Datatypes}{\coqdocconstructor{None}}\coqdoceol
\coqdocindent{8.00em}
\coqdockw{end}\coqdoceol
\coqdocindent{1.00em}
\coqdockw{end}.\coqdoceol
\coqdocnoindent
\coqdockw{Fixpoint} \coqdef{fmcs a2 ans3.ext.progDenote}{progDenote}{\coqdocdefinition{progDenote}} (\coqdocvar{p} : \coqref{fmcs a2 ans3.ext.prog}{\coqdocdefinition{prog}}) (\coqdocvar{s} : \coqdocdefinition{stack}) : \coqexternalref{option}{http://coq.inria.fr/distrib/8.4pl5/stdlib/Coq.Init.Datatypes}{\coqdocinductive{option}} \coqdocdefinition{stack} :=\coqdoceol
\coqdocindent{1.00em}
\coqdockw{match} \coqdocvariable{p} \coqdockw{with}\coqdoceol
\coqdocindent{1.00em}
\ensuremath{|} \coqexternalref{nil}{http://coq.inria.fr/distrib/8.4pl5/stdlib/Coq.Init.Datatypes}{\coqdocconstructor{nil}} \ensuremath{\Rightarrow} \coqexternalref{Some}{http://coq.inria.fr/distrib/8.4pl5/stdlib/Coq.Init.Datatypes}{\coqdocconstructor{Some}} \coqdocvariable{s}\coqdoceol
\coqdocindent{1.00em}
\ensuremath{|} \coqdocvar{i} \coqexternalref{:list scope:x '::' x}{http://coq.inria.fr/distrib/8.4pl5/stdlib/Coq.Init.Datatypes}{\coqdocnotation{::}} \coqdocvar{p'} \ensuremath{\Rightarrow} \coqdockw{match} \coqref{fmcs a2 ans3.ext.instrDenote}{\coqdocdefinition{instrDenote}} \coqdocvar{i} \coqdocvariable{s} \coqdockw{with}\coqdoceol
\coqdocindent{7.50em}
\ensuremath{|} \coqexternalref{None}{http://coq.inria.fr/distrib/8.4pl5/stdlib/Coq.Init.Datatypes}{\coqdocconstructor{None}} \ensuremath{\Rightarrow} \coqexternalref{None}{http://coq.inria.fr/distrib/8.4pl5/stdlib/Coq.Init.Datatypes}{\coqdocconstructor{None}}\coqdoceol
\coqdocindent{7.50em}
\ensuremath{|} \coqexternalref{Some}{http://coq.inria.fr/distrib/8.4pl5/stdlib/Coq.Init.Datatypes}{\coqdocconstructor{Some}} \coqdocvar{s'} \ensuremath{\Rightarrow} \coqref{fmcs a2 ans3.progDenote}{\coqdocdefinition{progDenote}} \coqdocvar{p'} \coqdocvar{s'}\coqdoceol
\coqdocindent{7.50em}
\coqdockw{end}\coqdoceol
\coqdocindent{1.00em}
\coqdockw{end}.\coqdoceol
\coqdocnoindent
\coqdockw{Fixpoint} \coqdef{fmcs a2 ans3.ext.compile}{compile}{\coqdocdefinition{compile}} (\coqdocvar{e} : \coqref{fmcs a2 ans3.ext.exp}{\coqdocinductive{exp}}) : \coqref{fmcs a2 ans3.ext.prog}{\coqdocdefinition{prog}} :=\coqdoceol
\coqdocindent{1.00em}
\coqdockw{match} \coqdocvariable{e} \coqdockw{with}\coqdoceol
\coqdocindent{1.00em}
\ensuremath{|} \coqref{fmcs a2 ans3.ext.Const}{\coqdocconstructor{Const}} \coqdocvar{n} \ensuremath{\Rightarrow} \coqref{fmcs a2 ans3.ext.iConst}{\coqdocconstructor{iConst}} \coqdocvar{n} \coqexternalref{:list scope:x '::' x}{http://coq.inria.fr/distrib/8.4pl5/stdlib/Coq.Init.Datatypes}{\coqdocnotation{::}} \coqexternalref{nil}{http://coq.inria.fr/distrib/8.4pl5/stdlib/Coq.Init.Datatypes}{\coqdocconstructor{nil}}\coqdoceol
\coqdocindent{1.00em}
\ensuremath{|} \coqref{fmcs a2 ans3.ext.Binop}{\coqdocconstructor{Binop}} \coqdocvar{b} \coqdocvar{e1} \coqdocvar{e2} \ensuremath{\Rightarrow} \coqexternalref{:list scope:x '++' x}{http://coq.inria.fr/distrib/8.4pl5/stdlib/Coq.Init.Datatypes}{\coqdocnotation{(}}\coqref{fmcs a2 ans3.compile}{\coqdocdefinition{compile}} \coqdocvar{e2}\coqexternalref{:list scope:x '++' x}{http://coq.inria.fr/distrib/8.4pl5/stdlib/Coq.Init.Datatypes}{\coqdocnotation{)}} \coqexternalref{:list scope:x '++' x}{http://coq.inria.fr/distrib/8.4pl5/stdlib/Coq.Init.Datatypes}{\coqdocnotation{++}} \coqexternalref{:list scope:x '++' x}{http://coq.inria.fr/distrib/8.4pl5/stdlib/Coq.Init.Datatypes}{\coqdocnotation{(}}\coqref{fmcs a2 ans3.compile}{\coqdocdefinition{compile}} \coqdocvar{e1}\coqexternalref{:list scope:x '++' x}{http://coq.inria.fr/distrib/8.4pl5/stdlib/Coq.Init.Datatypes}{\coqdocnotation{)}} \coqexternalref{:list scope:x '++' x}{http://coq.inria.fr/distrib/8.4pl5/stdlib/Coq.Init.Datatypes}{\coqdocnotation{++}} \coqexternalref{:list scope:x '++' x}{http://coq.inria.fr/distrib/8.4pl5/stdlib/Coq.Init.Datatypes}{\coqdocnotation{(}}\coqref{fmcs a2 ans3.ext.iBinop}{\coqdocconstructor{iBinop}} \coqdocvar{b} \coqexternalref{:list scope:x '::' x}{http://coq.inria.fr/distrib/8.4pl5/stdlib/Coq.Init.Datatypes}{\coqdocnotation{::}} \coqexternalref{nil}{http://coq.inria.fr/distrib/8.4pl5/stdlib/Coq.Init.Datatypes}{\coqdocconstructor{nil}}\coqexternalref{:list scope:x '++' x}{http://coq.inria.fr/distrib/8.4pl5/stdlib/Coq.Init.Datatypes}{\coqdocnotation{)}}\coqdoceol
\coqdocindent{1.00em}
\coqdockw{end}.\coqdoceol
\coqdocemptyline
\coqdocnoindent
\coqdockw{End} \coqref{fmcs a2 ans3.ext}{\coqdocmodule{ext}}.\coqdoceol
\coqdocemptyline
\end{coqdoccode}
