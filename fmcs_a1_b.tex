\documentclass[a4paper,12pt]{article}

\usepackage[utf8]{inputenc}
\usepackage{graphicx}
\usepackage{xcolor}

%\usepackage[defaultmono]{droidmono}

\usepackage{amsmath,amssymb,amsthm,textcomp}
\usepackage{enumerate}
\usepackage{multicol}
\usepackage{tikz}

\usepackage{geometry}
\geometry{total={210mm,297mm},
left=25mm,right=25mm,%
bindingoffset=0mm, top=20mm,bottom=20mm}

% code listing settings
\usepackage{listings}
%%%----------%%%----------%%%----------%%%----------%%%

\newcommand*{\quoteTitle}[1]{{#1}\ignorespaces}%
\newenvironment{Quote}[1]{
    \medskip\par\noindent\quoteTitle{#1}
    \par\noindent
    \begin{quote}
    }{
    \end{quote}
    \par\noindent\ignorespacesafterend
}

\newtheorem{theorem}{Theorem}

\begin{document}
\title{Fundamentals of MCS: CS Assignment 1 }

\author{Afzal Naveed - 15M54074}

\date{\today}

\maketitle

\section*{Q1.1. Equivalent classes}
\noindent

\paragraph{}A language $L$ is regular if and only if $\Sigma^{*}$ is divisible to a finite number of equivalent classes by $\approx_L$. Assume that we have a regular language $L$, which is recognized by some Deterministic Finite Automaton M. Since the number of states in M is finite, we has a finite set of states of M. Suppose a string input $l \in \Sigma^{*}$, decompose this string in to u and v: $l = uv$. Call $S_u$ as the state of the DFA after input $u$, we have:
$$ \forall u, w \in \Sigma^{*} \mbox{ that makes M ends up in state } S_u : \partial_{u}L = \partial_{v}L = \{v\} : M \mbox{ accepts } uv \mbox{ and } wv $$ 
because both $u$ and $w$ makes M ends up in same state, due to the determinism, there are no finite length string that can distinguish $u$ and $w$. Therefore, if we create sets of strings that makes the automaton ends up in each state, we will have a finite set of equivalent relation.
\paragraph{}On the other hand, suppose we have a language $L$ that has a finite set of equivalent classes. We can construct a DFA by creating each state for each equivalent classes. In each state, there always is one and only one next state transition for a given input (because \emph{any} string $\in \Sigma^{*}$ belongs to a class). The start state is the class of empty string and the end states is class such that:
$$ \partial_uL = \partial_vL = {\epsilon} $$
Therefore, we can construct a DFA recognizes language $L$, given a finite set of equivalent classes of $\Sigma^{*}$. Hence, we have the statement: ``A language $L$ is regular if and only if $\Sigma^{*}$ is divisible to a finite number of equivalent classes by $\approx_L$.

\section*{Q2.2. NP $\subseteq$ EXP}
\noindent
By definition, the EXPTIME class contains all \emph{solvable} problems using a deterministic Turing machine, which runs in exponential time.
$$ \mbox{EXP} = \bigcup_{k \in N} \mbox{DTIME} (2^{n^k}) $$
The NP class is defined by the term of non-deterministic Turing machine:
$$ \mbox{NP} = \bigcup_{k \in N} \mbox{NTIME} (n^k) $$
An alternative definition of the NP class is stated as follow:
$$ \mbox{Problem } L \mbox{ is in NP if } \exists \mbox{ a Turing Machine M s.t.\ for any } x \in \{0,1\}^{*}$$
$$ \ \ \ \ \ \ x \in L \Leftrightarrow \exists u \in \{0,1\}^{p(|x|)} \mbox{ s.t } M(x,u) = 1$$
where M is a polynomial-time Turing Machine and $p(|x|)$ is a polynomial. $u$ is called a certificate for $x$. This means that a NP class problem is a problem that we can verify the correctness of an answer for the problem given some certificate in polynomial time. 
\paragraph{}Suppose we have a problem $L \in$ NP, we know that there is a polynomial-time Turing Machine s.t.
\begin{itemize}
    \item If $x \notin L \mbox{, then } \forall u : M(x,u) = 0$. The Turing Machine rejects all input.
    \item If $x \in L \mbox{, then } \exists u : M(x,u) = 1$. The Turing Machine accepts some input.
\end{itemize}
For such problem, one simple approach is using brute-force search through all possible $u$ for some input $x$. Since our Turing Machine in this case is polynomial, the time it needs to verify each $u$ is $O(n^k)$. The total number of $u$ in this case is: 
$$ |\{u\}| = O(2^{p(|x|)}) $$
Therfore, the time it takes for the Turing Machine to search for all $u$ with some given input x is: 
$$ O(2^{p\{|x|\}}) \times O(n^k) = O(2^{n^k}) \subseteq \mbox{DTIME}(2^{n^k}) = \mbox{EXP} $$
Hence, \textbf{NP $\boldsymbol{\subseteq}$ EXP.} 
\paragraph{}A NP problem that does not seem to belong Time($2^{l^2})$ is some problem that takes $p(|x|) = l^k, k \geq 3$ steps to verify one instance of cerificate. Such problem does not seem to be bounded by Time($2^{l^2})$, but by some higher order time complexity. One example is k-D tensor fractorization. Suppose a tensor M is given and we want to find all pair of tensors that has product equals M. This computational task does not seem to belong Time($2^{l^2})$).

\end{document}
