%%% Template originaly created by Karol Kozioł (mail@karol-koziol.net) and modified for ShareLaTeX use

\documentclass[a4paper,12pt]{article}

\usepackage[utf8]{inputenc}
\usepackage{graphicx}
\usepackage{xcolor}

\usepackage{tgheros}
%\usepackage[defaultmono]{droidmono}

\usepackage{amsmath,amssymb,amsthm,textcomp}
\usepackage{enumerate}
\usepackage{multicol}
\usepackage{tikz}

\usepackage{geometry}
\geometry{total={210mm,297mm},
left=25mm,right=25mm,%
bindingoffset=0mm, top=20mm,bottom=20mm}


\linespread{1.3}

\newcommand{\linia}{\rule{\linewidth}{0.5pt}}

% custom theorems if needed
% my own titles
\makeatletter
\renewcommand{\maketitle}{
\begin{center}
\vspace{2ex}
{\huge \textsc{\@title}}
\vspace{1ex}
\\
\linia\\
\@author \hfill \@date
\vspace{4ex}
\end{center}
}
\makeatother
%%%

% custom footers and headers
\usepackage{fancyhdr}
\pagestyle{fancy}
\lhead{}
\chead{}
\rhead{}
\lfoot{SINA: Problem 0}
\cfoot{Page \thepage}
\rfoot{Hoang NT}
\renewcommand{\headrulewidth}{0pt}
\renewcommand{\footrulewidth}{0pt}
%

% code listing settings
\usepackage{listings}
\lstset{
    language=Python,
    basicstyle=\ttfamily\small,
    aboveskip={1.0\baselineskip},
    belowskip={1.0\baselineskip},
    columns=fixed,
    extendedchars=true,
    breaklines=true,
    tabsize=4,
    prebreak=\raisebox{0ex}[0ex][0ex]{\ensuremath{\hookleftarrow}},
    frame=lines,
    showtabs=false,
    showspaces=false,
    showstringspaces=false,
    keywordstyle=\color[rgb]{0.627,0.126,0.941},
    commentstyle=\color[rgb]{0.133,0.545,0.133},
    stringstyle=\color[rgb]{01,0,0},
    numbers=left,
    numberstyle=\small,
    stepnumber=1,
    numbersep=10pt,
    captionpos=t,
    escapeinside={\%*}{*)}
}

%%%----------%%%----------%%%----------%%%----------%%%

\begin{document}

\title{Social and Information Network Analysis: Problem 0}

\author{NGUYEN T. Hoang}

\date{2015-10-20} 

\maketitle


\section*{Problem 0.1}

This data was derived from comments on recipes on the Allrecipes website. An edge exists between ingredient \emph{e} and ingredient \emph{j}, if \emph{j} was recommended as a substitute for i in at least 5\% of the comments recommending substitutions, e.g. ``Great recipe, but I used Brussels sprouts instead of broccoli\ldots''\\

\noindent
The network and the name for each nodes are contained in 2 files: \textbf{ingredient\_key.txt} and \textbf{ingredient\_substitutes.txt}. \\ 

\section*{Questions and Answer}
\begin{enumerate}
    \item Find the size of the largest SCC. What percentage of the nodes is in the largest SCC? The largest strongly connected component is calculated in Snap.py with function \texttt{GetMxScc}.
\end{enumerate}

\noindent
\textbf{Answer:} The largest SCC of ingredient network has 244 nodes and it is 43\% of all nodes in the network.

\begin{lstlisting}[label={list:first},caption=Load the network from file name and return the size of the largest SCC]
import snap

IG = snap.LoadEdgeList(snap.PNGraph, "ingredient_substitutes.txt", 0, 1)
maxScc = snap.GetMxScc(IG)
snap.PrintInfo(maxScc)
# Return 244 nodes for the larges SCC.
# There is total of 562 nodes in the network.
\end{lstlisting}

\begin{enumerate}
    \item Find the ingredient most distant (via an undirected path within the WCC) from \emph{cocoa powder}. What is the distance? You could use the Snap.py function \texttt{GetNodesAtHop} to find nodes at a given distance.
\end{enumerate}

\end{document}
