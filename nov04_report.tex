%
% Montly report: October 2016
% Author: Hoang Nguyen
%
\documentclass[12pt,twoside]{article}

\usepackage{graphicx}
\usepackage{amsmath}
\usepackage{dsfont}

\input{macros}

\usepackage{color, colortbl}
\usepackage{longtable}
\definecolor{Gray}{gray}{0.8}

\setlength{\oddsidemargin}{0pt}
\setlength{\evensidemargin}{0pt}
\setlength{\textwidth}{6.5in}
\setlength{\topmargin}{0in}
\setlength{\textheight}{8.5in}

% Fill these in!
\newcommand{\theproblemsetnum}{1}
\newcommand{\releasedate}{November 01, 2016~~}
\newcommand{\partaduedate}{October 04}
\newcommand{\tabUnit}{3ex}
\newcommand{\tabT}{\hspace*{\tabUnit}}

\begin{document}

\handout{\textsc{Monthly Report - Nov 2016}}{\releasedate}

\newif\ifsolution
\solutiontrue
\newcommand{\solution}{\textbf{My plan:}}

\begin{enumerate}
  \item Research activities in October.
  \item Research plan in November.
\end{enumerate}

\setlength{\parindent}{0pt}

\medskip

\hrulefill

\textbf{Collaborators:}
%%% COLLABORATORS START %%%
Choong Jun Jin, Kaushalya, Sunil, Yann.
%%% COLLABORATORS END %%%

\vspace{1em}

\section{Research activities in October}

\begin{center}
  \renewcommand{\arraystretch}{1.5}
  \begin{longtable}{| c | p{6.5cm} | p{6.5cm} |}
  \hline
  & \textbf{\textsc{Toward Master Thesis}} & \textbf{\textsc{WSDM Cup 2017}} \\ \hline
  \textbf{\textsc{Topic}} & Submodularity and random processes in network. & 
  Vandalism detection and knowledge graph.\\ \hline
  \textbf{\textsc{Idea}} & Determinantal Point Processes (DPPs) and its estimation
  using simple submodular functions on sets of nodes or edges are an interesting
  research field. DPPs can be used as a effective initialization technique for
  non-convex MLE problems. On the other hand, submodularity functions on edges
  of a graph can be used to define a novel community detection criteria in
  addition to existing techniques (spectral clustering, modularity, ...).
  & Detect vandalism in Wikidata by
  neural random forest, or train a specialized neural network
  to detect edge-case vandalism output by the traditional random
  forest model. For triple scoring, I mined the Google's rank score
  for each person's name using Google Knowledge Graph API. By using
  the aforementioned Google's score and the similarity score
  learned by Skipgram model from Wikidata's text corpus, we train
  a simple feed-forward neural network to classify the popularity
  of each name-job or name-country on a scale of 0 to 7. \\ \hline
  \textbf{\textsc{Activities}} & For the submodularity and determinantal point
  processes, I am still in the literature research phrase. I also have received
  the review for the ``Motif-Aware Graph Embedding'' paper. I am rewriting the
  paper as suggested by the reviewers.
  & I have assembled a team
  of 4 to participate
  in WSDM Cup 2017. This year WSDM Cup 2017 consists of 2 task:
  Vandalism detection on Wikidata and Triple scoring. Details
  are given in the reference.\\ \hline
  \end{longtable}
\end{center}

\section{Research plan in November}

\begin{center}
  \renewcommand{\arraystretch}{1.5}
  \begin{longtable}{| c | p{12cm} |}
  \hline
  & \textbf{\textsc{November - 2016}} \\ \hline
  \textbf{\textsc{Topic}} & Rare event detection, knowledge graph,
  and submodular models on graph.\\ \hline
  \textbf{\textsc{Plan}} & About the WSDM Cup 2017, we will finalize our approach
  and create a prototype for submission by the end of November. About my research,
  by the end of November, I will have substantial knowledge about random processes
  on graph and write a literature review.
  \\ \hline
  \end{longtable}
\end{center}

\end{document}
