%
% ART.T548 - Advanced artificial intelligence 
% Author: Hoang Nguyen
%
\documentclass[12pt,twoside]{article}

\usepackage{amsmath, amssymb}
\usepackage{graphicx}
\usepackage{dsfont}
\usepackage{wrapfig}
\usepackage{float}
\usepackage{subcaption}

\input{macros}

\setlength{\oddsidemargin}{0pt}
\setlength{\evensidemargin}{0pt}
\setlength{\textwidth}{6.5in}
\setlength{\topmargin}{0in}
\setlength{\textheight}{8.5in}

% Fill these in!
\newcommand{\theproblemsetnum}{3}
\newcommand{\releasedate}{Thursday, July 31, 2016}
\newcommand{\partaduedate}{Monday, August 8, 2016}
\newcommand{\tabUnit}{3ex}
\newcommand{\tabT}{\hspace*{\tabUnit}}

\makeatletter
\newcommand*{\indep}{%
\mathbin{%
\mathpalette{\@indep}{}%
}%
}
\newcommand*{\nindep}{%
\mathbin{%                   % The final symbol is a binary math operator
\mathpalette{\@indep}{\not}% \mathpalette helps for the adaptation
% of the symbol to the different math styles.
}%
}
\newcommand*{\@indep}[2]{%
% #1: math style
% #2: empty or \not
\sbox0{$#1\perp\m@th$}%        box 0 contains \perp symbol
\sbox2{$#1=$}%                 box 2 for the height of =
\sbox4{$#1\vcenter{}$}%        box 4 for the height of the math axis
\rlap{\copy0}%                 first \perp
\dimen@=\dimexpr\ht2-\ht4-.2pt\relax
% The equals symbol is centered around the math axis.
% The following equations are used to calculate the
% right shift of the second \perp:
% [1] ht(equals) - ht(math_axis) = line_width + 0.5 gap
% [2] right_shift(second_perp) = line_width + gap
% The line width is approximated by the default line width of 0.4pt
\kern\dimen@
{#2}%
% {\not} in case of \nindep;
% the braces convert the relational symbol \not to an ordinary
% math object without additional horizontal spacing.
\kern\dimen@
\copy0 %                       second \perp
} 
\makeatother

\begin{document}

\handout{Quiz 3 - Lecture 14 (Prof. Shinoda)}{\releasedate}

\newif\ifsolution
\solutiontrue
\newcommand{\solution}{\textbf{Solution:}}

\noindent

\begin{enumerate}
  \item Prove that $p(\mathbf{x}) = \sum_{k=1}^{K} \pi_k \mathcal{N}(\mathbf{x}|\boldsymbol \mu_k,\boldsymbol \Sigma_k)$
  \item Discuss the future prospect of deep learning and its related techniques.
\end{enumerate}

\setlength{\parindent}{0pt}

\hrulefill

\textbf{Collaborators:}
%%% COLLABORATORS START %%%
None.
%%% COLLABORATORS END %%%

\begin{exercises}

\problem \textbf{Prove that} $p(\mathbf{x}) = \sum_{k=1}^{K} \pi_k \mathcal{N}(\mathbf{x}|\boldsymbol \mu_k,\boldsymbol \Sigma_k)$

\ifsolution \solution{}
%%% Ex1 SOLUTION START %%%
By definition, $\mathbf{z}$ is one-hot encoding representation, we have:

\begin{equation*}
  \begin{aligned}
    p(\mathbf{z}) & = \prod^{K}_{k=1} \pi_k^{z_k} \\
    p(\mathbf{x}|\mathbf{z}) & = \prod^{K}_{k=1} \mathcal{N} (\mathbf{x}|\boldsymbol \mu_k, \boldsymbol \Sigma_k)^{z_k}
  \end{aligned}
\end{equation*}

By the product rule, we have the join probability of $\mathbf{x}$ and $\mathbf{z}$ as follow:
$$ p(\mathbf{x}, \mathbf{z}) = p(\mathbf{x}|\mathbf{z}) p(\mathbf{z}) $$

Using the sum product to compute the marginal $p(\mathbf{x})$:
\begin{equation*}
  \begin{aligned}
    p(\mathbf{x}) & = \sum_{\mathbf{z}} p(\mathbf{x}|\mathbf{z}) p(\mathbf{z})\\
                  & = \sum_{\mathbf{z}} \prod^{K}_{k=1} \mathcal{N} (\mathbf{x}|\boldsymbol \mu_k, \boldsymbol \Sigma_k)^{z_k} \pi_k^{z_k}
                  & = \sum_{j=1}^{K} \prod^{K}_{k=1} (\pi_k \mathcal{N}(\mathbf{x}|\boldsymbol \mu_k, \boldsymbol \Sigma_k)^{\delta_{jk}},
  \end{aligned}
\end{equation*}
where $\delta_{jk}$ is the Kronecker delta. Simply rewrite the product keeping not-1 values, we have the desired result:

$$ p(\mathbf{x}) = \sum_{k=1}^{K} \pi_k \mathcal{N}(\mathbf{x}|\boldsymbol \mu_k,\boldsymbol \Sigma_k) $$

%%% Ex1 SOLUTION END %%%
\fi

\problem \textbf{Discuss the future prospect of deep learning and related techniques.}

\ifsolution \solution{}
%%% PROBLEM 2(a) SOLUTION START %%
The area in deep learning research that I am interested in are active learning
and multi-task learning. Active learning deals with problem of diversity in 
the training data. 

%%% PROBLEM 2(a) SOLUTION END %%%
\fi


\end{exercises}

\end{document}
