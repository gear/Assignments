%
% ART.T548 - Advanced artificial intelligence 
% Author: Hoang Nguyen
%
\documentclass[12pt,twoside]{article}

\usepackage{amsmath, amssymb}
\usepackage{graphicx}
\usepackage{dsfont}
\usepackage{wrapfig}
\usepackage{float}
\usepackage{subcaption}

\input{macros}

\setlength{\oddsidemargin}{0pt}
\setlength{\evensidemargin}{0pt}
\setlength{\textwidth}{6.5in}
\setlength{\topmargin}{0in}
\setlength{\textheight}{8.5in}

% Fill these in!
\newcommand{\theproblemsetnum}{2}
\newcommand{\releasedate}{Thursday, July 28, 2016}
\newcommand{\partaduedate}{Monday, August 1, 2016}
\newcommand{\tabUnit}{3ex}
\newcommand{\tabT}{\hspace*{\tabUnit}}

\makeatletter
\newcommand*{\indep}{%
\mathbin{%
\mathpalette{\@indep}{}%
}%
}
\newcommand*{\nindep}{%
\mathbin{%                   % The final symbol is a binary math operator
\mathpalette{\@indep}{\not}% \mathpalette helps for the adaptation
% of the symbol to the different math styles.
}%
}
\newcommand*{\@indep}[2]{%
% #1: math style
% #2: empty or \not
\sbox0{$#1\perp\m@th$}%        box 0 contains \perp symbol
\sbox2{$#1=$}%                 box 2 for the height of =
\sbox4{$#1\vcenter{}$}%        box 4 for the height of the math axis
\rlap{\copy0}%                 first \perp
\dimen@=\dimexpr\ht2-\ht4-.2pt\relax
% The equals symbol is centered around the math axis.
% The following equations are used to calculate the
% right shift of the second \perp:
% [1] ht(equals) - ht(math_axis) = line_width + 0.5 gap
% [2] right_shift(second_perp) = line_width + gap
% The line width is approximated by the default line width of 0.4pt
\kern\dimen@
{#2}%
% {\not} in case of \nindep;
% the braces convert the relational symbol \not to an ordinary
% math object without additional horizontal spacing.
\kern\dimen@
\copy0 %                       second \perp
} 
\makeatother

\begin{document}

\handout{Quiz 2 - Lecture 13 (Prof. Shinoda)}{\releasedate}

\newif\ifsolution
\solutiontrue
\newcommand{\solution}{\textbf{Solution:}}
\begin{wraptable}[8]{l}{0.3\textwidth}
  \centering
  \begin{tabular}{ccc|c}
    a & b & c & P(a,b,c) \\ \hline
    0 & 0 & 0 & 0.192 \\
    0 & 0 & 1 & 0.144 \\
    0 & 1 & 0 & 0.048 \\
    0 & 1 & 1 & 0.216 \\
    1 & 0 & 0 & 0.192 \\
    1 & 0 & 1 & 0.064 \\
    1 & 1 & 0 & 0.048 \\
    1 & 1 & 1 & 0.096 \\
  \end{tabular}
  \caption{Join probability}
\end{wraptable}

\hspace{2em}

\noindent
Consider the probability table of random variables $a$, $b$, and $c$
given in Table 1. Answer these following questions: 

\begin{enumerate}
  \item Prove that $a \nindep b$.
  \item Prove that $ a \indep b \ | \ c$.
  \item Show $P(a,b,c) = P(a)P(c \ | \ a)P(b \ | \ c)$.
  \item Illustrate a DAG correspondes to question 3.
\end{enumerate}

\setlength{\parindent}{0pt}

\hspace{1em}

\hrulefill

\textbf{Collaborators:}
%%% COLLABORATORS START %%%
None.
%%% COLLABORATORS END %%%

\begin{exercises}

\problem \textbf{Prove $a \nindep b$}

\ifsolution \solution{}
%%% Ex1 SOLUTION START %%%
Assume: $a \indep b \Leftrightarrow P(a,b) = P(a)P(b)$. We have a counter example:

\begin{equation*}
  \begin{aligned}
    P(b=0) & = \sum_{a,c} P(a,b=0,c) = 0.192+0.144+0.192+0.064=0.592 \\
    P(a=0) & = \sum_{b,c} P(a=0,b,c) = 0.192+0.064+0.048+0.096=0.4 \\
    P(a=0,b=0) & = \sum_{c} P(a=0,b=0,c) = 0.192+0.144 = 0.336 \\
               & \neq P(a=0)P(b=0) = 0.4 \times 0.592 = 0.2368
  \end{aligned}
\end{equation*}

Therefore, $a \nindep b$.
%%% Ex1 SOLUTION END %%%
\fi

\problem \textbf{Prove $a \indep b \ | \ c$}

\ifsolution \solution{}
%%% PROBLEM 2(a) SOLUTION START %%%
We have the conditional probabilities:
\begin{equation*}
  \begin{aligned}
    P(a,b \ | \ c) & = \frac{P(a,b,c)}{P(c)} \\
    P(a \ | \ c) & = \frac{\sum_{b}P(a,b,c)}{P(c)} \\
    P(b \ | \ c) & = \frac{\sum_{a}P(a,b,c)}{P(c)} \\
  \end{aligned}
\end{equation*}

We also have:
\begin{equation*}
  \begin{aligned}
    P(c=1) & = \sum_{a,b} P(a,b,c=1) = 0.216+0.144+0.064+0.096=0.52 \\
    P(c=0) & = \sum_{b,b} P(a,b,c=1) = 1-0.52 = 0.48 \\
  \end{aligned}
\end{equation*}

We have table of join probability conditioned on $c$:
\begin{table}[h!]
  \centering
  \begin{tabular}{ccc|cc}
    a & b & c & $P(a,b \ | \ c)$ & $P(a \ | \ c) P(b \ | \ c)$ \\ \hline
    0 & 0 & 0 & 0.4 & 0.4   \\
    0 & 0 & 1 & 0.277 & 0.277  \\
    0 & 1 & 0 & 0.1 & 0.1  \\
    0 & 1 & 1 & 0.415 & 0.415 \\
    1 & 0 & 0 & 0.4 & 0.4 \\
    1 & 0 & 1 & 0.123 & 0.123 \\
    1 & 1 & 0 & 0.1 & 0.1 \\
    1 & 1 & 1 & 0.185 & 0.185 \\
  \end{tabular}
  \caption{Join probability conditioned on c}
\end{table}

From Table 2, we have:
$$ P(a,b \ | \ c) = P(a \ | \ c)P(b \ | \ c) $$

Therefore, $a \indep c \ | \ c$.
%%% PROBLEM 2(a) SOLUTION END %%%
\fi

\problem \textbf{Show $P(a,b,c) = P(a)P(c \ | \ a)P(b \ | \ c)$}

\ifsolution \solution{}
Using sum rule similar to the previous questions, we have these following
probability tables:

\begin{table}[hb!]
  \begin{subtable}{.3\linewidth}
    \centering
      \begin{tabular}{c|c}
        a & $P(a)$ \\ \hline
        0 & 0.6 \\
        1 & 0.4 \\
      \end{tabular}
      \caption{$P(a)$}
  \end{subtable}
  %
  \begin{subtable}{.3\linewidth}
    \centering
      \begin{tabular}{cc|c}
        c & a & $P(c \ | \ a)$ \\ \hline
        0 & 0 & 0.4 \\
        0 & 1 & 0.6 \\
        1 & 0 & 0.6 \\
        1 & 1 & 0.4 \\
      \end{tabular}
      \caption{$P(a)$}
  \end{subtable}
  %
  \begin{subtable}{.3\linewidth}
    \centering
      \begin{tabular}{cc|c}
        b & c & $P(b \ | \ c)$ \\ \hline
        0 & 0 & 0.8 \\
        0 & 1 & 0.4 \\
        1 & 0 & 0.2 \\
        1 & 1 & 0.6 \\
      \end{tabular}
      \caption{$P(a)$}
  \end{subtable}

  \caption{Conditional probability tables}
\end{table}

Multiply the probabilities in Table 3 gives us the result:
$$ P(a,b,c) = P(a)P(c \ | \ a)P(b \ | \ c) $$
\fi

\pagebreak

\problem \textbf{Illustrate a DAD corresponds to question 3}

\ifsolution \solution{}
\begin{figure}[h!]
  \centering
  \includegraphics[width=0.5\textwidth]{dag}
  \caption{DAG corresponding to question 3.}
\end{figure}

\fi

\end{exercises}

\end{document}
