\documentclass[a4paper,12pt]{article}

\usepackage[utf8]{inputenc}
\usepackage{graphicx}
\usepackage{xcolor}

%\usepackage[defaultmono]{droidmono}

\usepackage{amsmath,amssymb,amsthm,textcomp}
\usepackage{enumerate}
\usepackage{multicol}
\usepackage{tikz}

\usepackage{geometry}
\geometry{total={210mm,297mm},
left=25mm,right=25mm,%
bindingoffset=0mm, top=20mm,bottom=20mm}


\linespread{1.3}

\newcommand{\linia}{\rule{\linewidth}{0.5pt}}

% custom theorems if needed
% my own titles
\makeatletter
\renewcommand{\maketitle} {
\begin{center}
\vspace{2ex}
{\huge \textsc{\@title}}
\vspace{1ex}
\\
\linia\\
\@author \hfill \@date
\vspace{4ex}
\end{center}
}
\makeatother
%%%

% custom footers and headers
\usepackage{fancyhdr}
\pagestyle{fancy}
\lhead{}
\chead{}
\rhead{}
\lfoot{DAGS - Part I}
\cfoot{Page \thepage}
\rfoot{15M54097}
\renewcommand{\headrulewidth}{0pt}
\renewcommand{\footrulewidth}{0pt}
%

% code listing settings
\usepackage{listings}
%%%----------%%%----------%%%----------%%%----------%%%

\newcommand*{\quoteTitle}[1]{{#1}\ignorespaces}%
\newenvironment{Quote}[1]{
    \medskip\par\noindent\quoteTitle{#1}
    \par\noindent
    \begin{quote}
    }{
    \end{quote}
    \par\noindent\ignorespacesafterend
}

\newtheorem{theorem}{Theorem}[section]
\newtheorem{corollary}{Corollary}[theorem]
\newtheorem{lemma}[theorem]{Lemma}
 

\begin{document}
\bibliographystyle{acm}
\title{DAGS - Part I Final Report}

\author{NGUYEN T. Hoang - SID: 15M54097}

\date{Spring 2016, W832 Tue-Fri Period 5-6 \\ \hfill Due date: 2015/06/09}

\maketitle

\vspace{10em}
\section*{Problem}
\noindent
Graph has always been an important model for real world interactions. With development of the internet, nowaday researchers refer to a large interconnected model with information flow (interactions) as a complex network. Inspired by the hyperbolic space and Albert Eistein's work on relativity, many researches have hinted that there is a hyperbolic structure underlies these complex networks. In this final report, I will summary some of the recent developments in hyperbolic geometry
of complex networks, focusing on navigability and $\delta$-hyperbolicity.

\vfill
\pagebreak
\section*{Answer:}

\section{Introduction}
There are many computation or scientific problems that are simplifed after their hyperbolic structure is identified \cite{2} \cite{3}. The most famous example probably is the hyperbolid model used in special relativity. Nowaday, in the 21st century, along with the development of social networks and internet of things, raise a more than ever important task of finding a model for a complex network. Inspired from the success of hyperbolic space and many hypotheses about the hidden fractal
geometry underlies our universe, complex network researchers have shifted their attention toward hyperbolic geometry.
\paragraph{Complex Network.} Complex network is defined as a large graph with set of \emph{edges} and \emph{vertices}. Each vertex in a complex network can be heretogeneous object representing a physical model. Each edge encodes the pairwise interactions between vertices. Many important structures can be modeled as complex network: Social network, functional brain network, biological metabolic networks, genome networks, etc. Within these real-world networks, researchers have found
that they share some common properties:

\begin{itemize}
    \item Small-world phenomenon: network diameter is extremely small comparing to its size.
    \item Power law degree distribution: Degree distribution (number of connected edges to a vertex) in a network follows a power law.
    \item Navigability: Using only local information, we can navigate from a node to any destination taking a relatively short path. \cite{1}
    \item High clustering coefficient: Vertices often belongs to a densely connected community.
\end{itemize}

A simple undirected graph is notated by $G = (V,E)$. This graph is naturally a discrete metric space $(V,d)$ with distance between vertices $d$. $(V,d)$ can be extended to a geodesic metric space withe very edge interpreted as a segment of length 1.

\paragraph{$\delta$-hyperbolicity.} Hyperbolic geometry has negative curvature and a generalization in the context of metric space. 

\begin{Quote}{Four-point condition - $\delta$-hyperbolic \cite{13}}
    \setlength{\parskip}{0em}
    In a metric psace $(X,d)$, given $u,v,w,x$ with $d(u,v) + d(w,x) \geq d(u,x) + d(w,v) \geq d(u,w) + d(v,x)$ in $X$, we note $\delta (u,v,w,x) = d(u,v) + d(w,x) - d(u,x) - d(w,v) $. $(X,d)$ is called $\delta$-hyperbolic if for any four points $u, v, w, x  \in X$, $\delta (u,v,w,x) \leq 2 \delta$. We note $\delta(X,d)$ the smallest possible value of such $\delta$, if ever exists.  \end{Quote}

\begin{Quote}{Rips condition \cite{12}}
    \setlength{\parskip}{0em}
    In a geodesic metric space $(X,d)$, given $u, v, w$ in $X$, we note $\Delta (u, v, w) = [u,v] \cup [v,w] \cup [w,u]$ a geodesic triangle. $[u,v], [v,w], [w, u]$ are called sides of $\Delta(u, v, w)$. A geodesic triangle is called $\delta$-slim if any point on a side is within distance $\delta$ to the union of other two sides. $(X,d)$ is called Rips $\delta$-hyperbolic if every geodesic triangle is $\delta$-slim. We note $\delta_{Rips} (X,d)$ the smallest possible value of such
    $\delta$, if ever exists.
\end{Quote}

On continuous metric spaces, $\delta$-hyperbolicity is a measure of negative curvature. On graphs, $\delta$-hyperbolicity is similar to a tree measure. Trees and block graphs are $0$-hyperbolic, while cycles $C_n$ are $O(n)$-hyperbolic \cite{23}. The benefits of embedding a graph into a hyperbolic space rather than an Euclidean space are proven theorecially and empirically in many researches. In summary, when $\delta(G)$ is bounded, problems such as graph diameter approximation, exact
center computing, and distance approximating tree in graph become tractible \cite{6,7,11}. In addition, \cite{21} had shown that a complex network such as the Internet embeds better in to a hyperbolic space than in to an Euclidean space. Since $\delta$-hyperbolicity reflects hyperbolic geometry of graphs, there might be connection between this metric and the behavior of complex network \cite{1}.

\section{Properties of $\delta$-hyperbolicity}
In this sections, I present some proof regarding the properties of $\delta$=hyperbolicity in graph. The diameter of a graph $G$ gives an upper bound of $\delta(G)$ \cite{1}.
\begin{lemma}
    For any graph $G$, $\delta(G) \leq \mbox{diam}(G) / 2$.
\end{lemma}
\begin{proof}
    There exists $u,v,w,x \in X$ with $d(u,v) + d(w,x) \geq d(u,x) + d(w,v) \geq d(u,w) + d(v,x)$ s.t. $2\delta(G) = d(u,v) + d(w,x) - d(u,x) - d(w,v)$.
    Moreover, $2\delta(G) \leq d(u,v) + d(w,x) - (d(u,x) + d(x,v) + d(u,w) + d(w,x)) / 2 \leq 2\mbox{min}(d(u,v), d(w,x)) \leq 2\mbox{diam}(G)$.
\end{proof}

\begin{Quote}{Glue sum of graphs}
    \setlength{\parskip}{0em}
    Let $G_1 = (V_1, E_1), G_2 = (V_2, E_2), \ldots, G_k = (V_k, E_k)$ be graphs such that, there is a metric space $(X,d)$ isometric to some $V_i' \subseteq V_i$ for all $G_i$. We note the glue sum $(G_1 \oplus \ldots \oplus G_k)|_X$ the graph formed by identifying all $V_i'$ with $X$ according to the isometries. Each $G_i$ í called a component of $G$.
\end{Quote}

\begin{lemma}
    We define $\tilde{\delta}_{Rips}^{X}(G)$ be the maximum of $\delta_{Rips}(G \cup <u,v>)$ for any $u, v \in X$ where $<u,v>$ is a segment of length $d(u,v)$ attached between $u,v$.
    Let $G = (G_1 \oplus \ldots \oplus G_k)|_X$, we have the following:
    $$ \delta_{Rips}(G) \leq \max_{1 \leq i \leq k}(\tilde{\delta}_{Rips}^X) + \mbox{diam}(X) $$
\end{lemma}
Proof for this lemma is provided in \cite{1}.

\begin{corollary}
    If $G = (G_1 \oplus \ldots \oplus G_k)|_X$ with $X$ a singleton, we have:
    $$\delta_{Rips}(G) \leq \max_{1 \leq i \leq k}(\delta_{Rips}(G_i)) $$
\end{corollary}

\begin{proof}
    We apply directly Lemma 2.2 by noticing that X is a singleton, so $\mbox{diam}(X) = 0$, and $\tilde{\delta}_{Rips}^X (G_i) = \delta_{Rips} (G_i)$.
\end{proof}

\section{$\delta$-hyperbolicity and navigability}
In this section, I will partially summary the analysis made in \cite{1} for $\delta$-hyperbolicty measurement of different network models including Kleinberg's model and Fan Chung's power-law graph.

\subsection{Kleinberg's model}
In this model, $n$ vertices form a square lattice on a plane, neighboring vertices are connected. Each vertex can have one long jump onto a random vertex chosen with probability proportional to $r^{-\alpha}$, where $r$ is the distance between these two nodes on the lattice and $\alpha$ is a parameter. In \cite{14}, Kleinberg proved that, when based on a two-dimensional lattice, with a greedy routing scheme using only local information, we can reach from one node to any other passing by
$O(\log^2 n)$ edges only when $\alpha = 2$. When based on a $d$-dimensional lattice, this model has good navigability when $\alpha = d$. Denote $K(n,d,\alpha)$ as the graph of Kleinberg's model based on $d$-dimensional lattice with parameter $\alpha$.

\begin{theorem}
    For $d$ fixed, with high probability with respect to $n$,
    $$\forall \epsilon > 0, \delta(K(n,d,d)) \geq (\log n)^{1/(1.5(d+1}+\epsilon)$$
    $\delta(K(n,d,\alpha)) = \Omega(\log n)$ when $0 \leq \alpha d$.
\end{theorem}

In this model, the authors show that extremely good $delta$-hyperbolicity is not required for good navigability.

\subsection{Fan Chung's power-law graph}
In Fan Chung's model \cite{8}, given number of vertices $n$, expected average degree $d$, expected maximum degree $m$ and exponent $\beta$, the network is constructed such that it has power-law degree distribution. The fraction of verticies has degree $d$ is proportional to $d^{-\beta}$. A mathematical formulation for this model seems non-trivial. However, empirical result from \cite{1} shows that there is a slight correlation between $\delta$-hyperbolicity and the network
diameter, although this correlation is not very strong.

\section{Conclusion}
Although the computation of $\delta$-hyperbolicity for a large network still requires a significant amount of computation power, but this metric shows potential to explain that underlying hyperbolic geometry is respoinsible for good navigation properties of some complex networks. The Kleinberg's model is a model with good navigability and the authors in \cite{1} have proved analytically an lower bound of $\delta(G)$ for this model. On the other hand, Fan Chung's power-law graph model
remains a challenge. In conclusion, the pursue of identifying hyperbolic structure in complex network is a challenging active research area. In this reserach, the authors have investigated the $\delta$-hyperbolicity of some theoretical network models. Although the result is not very clear that $\delta$-hyperbolicity is a good measure of navigability, this research has contributed to the study of hyperbolic structure of complex network.
\bibliography{dg_final}

\end{document}
