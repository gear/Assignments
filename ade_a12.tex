\documentclass[a4paper,12pt]{article}

\usepackage[utf8]{inputenc}
\usepackage{graphicx}
\usepackage{xcolor}

%\usepackage{tgheros}
%\usepackage[defaultmono]{droidmono}

\usepackage{amsmath,amssymb,amsthm,textcomp}
\usepackage{enumerate}
\usepackage{enumitem}
\usepackage{multicol}
\usepackage{tikz}
\usetikzlibrary{arrows,calc}
\tikzset{
 >=stealth',
 help lines/.style={dashed, thick},
 axis/.style={<->},
 important line/.style={thick},
 connection/.style={thick, dotted},
}

\usepackage{CJKutf8}

\usepackage{geometry}
\geometry{total={210mm,297mm},
left=25mm,right=25mm,%
bindingoffset=0mm, top=20mm,bottom=20mm}


\linespread{1.3}

\newcommand{\linia}{\rule{\linewidth}{0.5pt}}

% custom theorems if needed
% my own titles
\makeatletter
\renewcommand{\maketitle} {
\begin{center}
\vspace{2ex}
{\LARGE \textsc{\@title}}
\vspace{1ex}
\\
\linia\\
\@author \hfill \@date
\vspace{4ex}
\end{center}
}
\makeatother
%%%

% custom footers and headers
\usepackage{fancyhdr}
\pagestyle{fancy}
\lhead{}
\chead{}
\rhead{}
\lfoot{ADE:Assignment 12}
\cfoot{Page \thepage}
\rfoot{15M54097}
\renewcommand{\headrulewidth}{0pt}
\renewcommand{\footrulewidth}{0pt}
%


% code listing settings
\usepackage{listings}
\lstset{
    language=SQL,
    basicstyle=\ttfamily\small,
    aboveskip={1.0\baselineskip},
    belowskip={1.0\baselineskip},
    columns=fixed,
    extendedchars=true,
    breaklines=true,
    tabsize=4,
    prebreak=\raisebox{0ex}[0ex][0ex]{\ensuremath{\hookleftarrow}},
    frame=lines,
    showtabs=false,
    showspaces=false,
    showstringspaces=false,
    keywordstyle=\color[rgb]{0.627,0.126,0.941},
    commentstyle=\color[rgb]{0.133,0.545,0.133},
    stringstyle=\color[rgb]{01,0,0},
    numbers=left,
    numberstyle=\small,
    stepnumber=1,
    numbersep=10pt,
    captionpos=t,
    escapeinside={\%*}{*)}
}


%\renewcommand*\thelstnumber{\arabic{lstnumber}:}

%\DeclareCaptionFormat{mylst}{\hrule#1#2#3}
%\captionsetup[lstlisting]{format=mylst, labelfont=bf, singlelinecheck=off, labelsep=space}

\begin{document}

\title{Advanced Data Engineering: Assign. 12}

\author{NGUYEN T. Hoang - SID: 15M54097}

\date{Fall 2015, W831 Tue. Period 5-6 \\ \hfill Due date: 2016/01/19}

\maketitle
\vspace{-4.5em}
\hspace{5.3em} (\begin{CJK}{UTF8}{min}ホアン\end{CJK})
\vspace{4em}
\section*{Problem}
\begin{itemize}
	\item Roughly estimate the execution time for the 4 distributed algorithms with the following assumptions:
\begin{itemize}
	\setlength{\parskip}{0cm}
	\setlength{\itemsep}{0cm}
	\item Cardinality of the relation {R} = 1,000,000; {S} = 500,000.
	\item Total length of a tuple: $st_R = 1,000B$; $st_S = 2,000B$.
	\item Disk transfer bandwidth: $B_{disk} = 10MB/s$.
	\item Network bandwidth: $B_{net} = 5MB/s$.
	\item Selectivity: $\alpha = 10 \%$, $\beta = 10 \%$, $\gamma = 10 \%$,  
	\item Hash Bit Vector: Use 1 bit for each tuple.
\end{itemize}
\end{itemize}
\section*{Question: Cost estimation for distributed algorithms.}
\setcounter{section}{1}

\textit{Estimate cost for Na\"{\i}ve, Semi-Join, 2-Way Semi-Join and Hashed Bit Vector distributed join algorithms.} 

\vspace{1.5em}
\noindent
\textbf{Answer:} 
\noindent
\paragraph{Na\"{\i}ve} According to the lecture note, the cost estimation for na\"{\i}ve algorithm is:
$$ \mathcal{C}_{\mbox{Na\"{\i}ve}} \approx \ \mathcal{C}_D(5S + 3R) + \mathcal{C}_C(S)$$
Since the data size of the relations are same, I assume relation S is sent. Replace given information about data, we have:
\begin{equation*}
	\begin{aligned}
		\mathcal{C}_{\mbox{Na\"{\i}ve}} \approx & \ \mathcal{C}_D(5S + 3R) + \mathcal{C}_C(S) \\
		= & \ \frac{5 \times \{S\} \times st_S + 3 \times \{R\} \times st_R}{B_{disk}} + \frac{\{S\} \times st_S}{B_{net}} \\
		= & \ \frac{5 \times 500,000 \times 2,000 + 3 \times 1,000,000 \times 1,000}{10,000,000} + \frac{500,000 \times 2,000}{5,000,000} \\
		= & \ 800 + 1000 = 1800\mbox{ seconds} = 30\mbox{ minutes}.
	\end{aligned}
\end{equation*}
\pagebreak

\paragraph{Semi-Join} According to the lecture note, the cost estimation for semi-join algorithm when we assume that projection can be overlapped on I/O is:
$$ \mathcal{C}_{\mbox{SJ}} \approx \ \mathcal{C}_D(4S + 5 \alpha S + 3R + 5 \beta R) + \mathcal{C}_C(\alpha S + \beta R)$$
Replace given information about data, we have:
\begin{equation*}
	\begin{aligned}
		\mathcal{C}_{\mbox{SJ}} \approx & \ \mathcal{C}_D(4S + 5 \alpha S + 3R + 5 \beta R) + \mathcal{C}_C(\alpha S + \beta R) \\
		= & \ \frac{(4 + 5\alpha) \times 500,000 \times 2,000 + (3 + 5 \beta) \times 1,000,000 \times 1,000}{10,000,000}  \\
		& + \frac{\alpha \times 500,000 \times 2,000 + \beta \times 1,000,000 \times 1,000}{5,000,000} \\
		= & \ 800 + 40 = 840\mbox{ seconds} = 14\mbox{ minutes}.
	\end{aligned}
\end{equation*}

\paragraph{2-Way Semi-Join} According to the lecture note, the cost estimation for 2-way semi-join algorithm is:
$$ \mathcal{C}_{\mbox{2WSI}} \approx \ \mathcal{C}_D(4R + 5 \beta R + 3 \alpha S + 2 \beta \gamma R + 2 \alpha \gamma S) + \mathcal{C}_C(\alpha \gamma S + \beta R)$$
Replace given information about data, we have:
\begin{equation*}
	\begin{aligned}
		\mathcal{C}_{\mbox{2WSI}} \approx & \ \mathcal{C}_D(4R + 5 \beta R + 3 \alpha S + 2 \beta \gamma R + 2 \alpha \gamma S) + \mathcal{C}_C(\alpha \gamma S + \beta R) \\
		= & \ \frac{(2 \alpha \gamma + 3 \alpha) \times 500,000 \times 2,000 + (4 + 5 \beta + 2 \beta \gamma) \times 1,000,000 \times 1,000}{10,000,000}  \\
		& + \frac{\alpha \gamma \times 500,000 \times 2,000 + \beta \times 1,000,000 \times 1,000}{5,000,000} \\
		= & \ 484 + 22 = 506\mbox{ seconds} = 8.4\mbox{ minutes}.
	\end{aligned}
\end{equation*}

\paragraph{Hashed Bit Vector} According to the lecture note, the cost estimation for hashed bit vector algorithm:
$$ \mathcal{C}_{\mbox{HB}} \approx \ \mathcal{C}_D(2S + 5 \alpha S + 4R + 4 \beta R) + \mathcal{C}_C( H_x + H_y + \alpha S)$$
Replace given information about data, we have:
\begin{equation*}
	\begin{aligned}
		\mathcal{C}_{\mbox{HB}} \approx & \ \mathcal{C}_D(2S + 5 \alpha S + 4R + 4 \beta R) + \mathcal{C}_C( H_x + H_y + \alpha S) \\
		= & \ \frac{(2 + 5\alpha) \times 500,000 \times 2,000 + (4 + 4 \beta) \times 1,000,000 \times 1,000}{10,000,000}  \\
		& + \frac{\alpha \times 500,000 \times 2,000 + 1,000,000 + 500,000}{5,000,000} \\
		= & \ 690 + 20.3 = 710.3\mbox{ seconds} = 11.8\mbox{ minutes}.
	\end{aligned}
\end{equation*}

\end{document}
