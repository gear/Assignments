%
% Montly report: October 2016
% Author: Hoang Nguyen
%
\documentclass[12pt,twoside]{article}

\usepackage{graphicx}
\usepackage{amsmath}
\usepackage{dsfont}

\input{macros}

\usepackage{color, colortbl}
\usepackage{longtable}
\definecolor{Gray}{gray}{0.8}

\setlength{\oddsidemargin}{0pt}
\setlength{\evensidemargin}{0pt}
\setlength{\textwidth}{6.5in}
\setlength{\topmargin}{0in}
\setlength{\textheight}{8.5in}

% Fill these in!
\newcommand{\theproblemsetnum}{1}
\newcommand{\releasedate}{October 06, 2016~~}
\newcommand{\partaduedate}{October 11}
\newcommand{\tabUnit}{3ex}
\newcommand{\tabT}{\hspace*{\tabUnit}}

\begin{document}

\handout{\textsc{Monthly Report - Oct 2016}}{\releasedate}

\newif\ifsolution
\solutiontrue
\newcommand{\solution}{\textbf{My plan:}}

\begin{enumerate}
  \item Research activities in August and September.
  \item Research plan in October.
  \item Problems in research (if any).
\end{enumerate}

\setlength{\parindent}{0pt}

\medskip

\hrulefill

\textbf{Collaborators:}
%%% COLLABORATORS START %%%
Choong Jun Jin, Kaushalya, Nukui, Sunil.
%%% COLLABORATORS END %%%

\vspace{1em}

\section{Research activities in August and September}

\begin{center}
  \renewcommand{\arraystretch}{1.5}
  \begin{longtable}{| c | p{6.5cm} | p{6.5cm} |}
  \hline
  & \textbf{\textsc{August - 2016}} & \textbf{\textsc{September - 2016}} \\ \hline
  \textbf{\textsc{Topic}} & Network embedding and motifs in network. & 
  Vandalism detection and knowledge graph.\\ \hline
  \textbf{\textsc{Idea}} & Use the graph's motif as a guide for the random
  walk in network embedding. The biased random walked generated
  by my model emphasizes on the local motif community of the network.
  Besides the motif walk, I also proposed the inverse motif walk
  to discover not-motif-community to use as negative samples.
  & Detect vandalism in Wikidata by
  neural random forest, or train a specialized neural network
  to detect edge-case vandalism output by the traditional random
  forest model. For triple scoring, I mined the Google's rank score
  for each person's name using Google Knowledge Graph API. By using
  the aforementioned Google's score and the similarity score
  learned by Skipgram model from Wikidata's text corpus, we train
  a simple feed-forward neural network to classify the popularity
  of each name-job or name-country on a scale of 0 to 7. \\ \hline
  \textbf{\textsc{Activities}} & I have conducted extensive experiments
  on BlogCatalog3 dataset (undirected triangle motif), and some 
  premilinary experiments on other datasets (bipartite and larger
  datasets). My model (named MAGE) shows superiority compared to 
  previous models. The reason for such performance lies at the
  number and quality of the negative samples obtained. I also
  wrote a paper and submitted to AAAI`17. The result for AAAI`17
  will be available on December 2016. & I have assembled a team
  of 4 (Nukui-san dropped out due to his own project) to participate
  in WSDM Cup 2017. This year WSDM Cup 2017 consists of 2 task:
  Vandalism detection on Wikidata and Triple scoring. Details
  are given in the reference.\\ \hline
  \textbf{\textsc{References}} & Deepwalk \cite{deepwalk}; 
  Planetoid \cite{planetoid}; LINE \cite{line}; node2vec \cite{n2v};
  MAGE source code \cite{mage}. & WSDM Cup 2017 \cite{wsdmcup}; 
  Vandalism detection \cite{van1}; Triple scoring \cite{trip1}. \\ \hline
  \end{longtable}
\end{center}

\section{Research plan in October}

\begin{center}
  \renewcommand{\arraystretch}{1.5}
  \begin{longtable}{| c | p{12cm} |}
  \hline
  & \textbf{\textsc{October - 2016}} \\ \hline
  \textbf{\textsc{Topic}} & Rare event detection, knowledge graph,
  and submodular models on graph.\\ \hline
  \textbf{\textsc{Plan}} & I will continue to develop my ideas for
  the WSDM Cup 2017 as mentioned in the previous seciton. October 15th,
  we will submit our first prototype of both systems (vandalism detection
  and triple scoring). On the other hand, I am studying about set theory
  and graphical submodular models because I want to conduct a concrete 
  mathematics proof for my motif walk model, which was submitted to
  AAAI`17 last month. Furthermore, I have great interest in submodularity
  and random processes, I plan to write my Master thesis on this topic.
  \\ \hline
  \textbf{\textsc{References}} & Submodularity \cite{submodularity}; 
  Submodularity in graph \cite{sgraph}; Matroid \cite{matroid}.
  \\ \hline
  \end{longtable}
\end{center}

\section{Problems}

\noindent
Currently I do not have any serious problem in my research. However,
There is a few minor problems that I will be grateful if you share
some of your comment if possible. 

\begin{itemize}
  \item Rare case prediction with machine learning. Currently, the
  vandalism takes about 7\% of all edit on Wikidata. The simple ZeroR
  model can easily achieve 93\% accuracy by predicting all edit as
  ``valid''. We have choosen the Random Forest (and its variant XGBoost)
  for this problem. However, we want to use some deep architecture
  (which performs very bad in this task). I wonder if you can give
  us some comment on this matter.
  \item Submodularity in set theory. I have a small wonder about
  how do you think about the future of this branch of discrete 
  mathematics. I want to write my master thesis on the amount of
  information (Fisher Information) gathered using some random
  process on a network. Do you think it is possible for me to
  pursue such mathematical topic? Please tell me if you prefer
  my master thesis to be more practical-oriented.

\end{itemize}

\textbf{Murata-sensei, thank you very much for your time!}

\medskip

\begin{thebibliography}{99}

\bibitem{deepwalk}
Perozzi, Bryan and Al-Rfou, Rami and Skiena, Steven. \textit{Deepwalk: Online learning of social representations}. Proceedings of the 20th ACM SIGKDD international conference on Knowledge discovery and data mining, 2014.

\bibitem{planetoid}
Zhilin Yang and William W. Cohen and Ruslan Salakhutdinov. \textit{Revisiting Semi-Supervised Learning with Graph Embeddings}. Proceedings of the 33rd International Conference on Machine Learning, 2016.

\bibitem{line}
Tang, Jian and Qu, Meng and Wang, Mingzhe and Zhang, Ming and Yan, Jun and Mei, Qiaozhu. \textit{Line: Large-scale information network embedding}. Proceedings of the 24th International Conference on World Wide Web, 2015.

\bibitem{n2v}
Grover, Aditya and Leskovec, Jure. \textit{node2vec: Scalable Feature Learning for Networks}. Proceedings of the 22nd ACM SIGKDD International Conference on Knowledge Discovery and Data Mining, 2016.

\bibitem{mage}
Nguyen, Hoang and Nukui, Shun and Murata, Tsuyoshi. \textit{https://github.com/anonsyuushi/mage}. Anonymous code submitted to AAAI`17.

\bibitem{wsdmcup}
WSDM Cup 2017. \textit{http://www.wsdm-cup-2017.org/}. Homepage for WSDM Cup 2017 containing two tasks: Vandalism dectection and Triple scoring.

\bibitem{van1}
Stefan Heindorf, Martin Potthast, Benno Stein, and Gregor Engels. \textit{Vandalism Detection in Wikidata}. In Proceedings of the 25th ACM International Conference on Information and Knowledge Management, 2016.

\bibitem{trip1}
Hannah Bast, Bjorn Buchhold, and Elmar HauBmann. \textit{Relevance Scores for Triples from Type-Like Relations}. In SIGIR, 2015.

\bibitem{submodularity}
Krause, Andreas, and Daniel Golovin. \textit{Submodular function maximization.}. Practical Approaches to Hard Problems 3.19 (2012): 8.

\bibitem{sgraph}
Frank, Andras. \textit{Submodular functions in graph theory}. Discrete Mathematics 111.1 (1993): 231-243.

\bibitem{matroid}
Oxley, James G. \textit{Matroid theory}. Vol 3. Oxford University Press, USA, 2006.

\end{thebibliography}

\end{document}
