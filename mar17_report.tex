%
% Montly report: March 2017
% Author: Hoang Nguyen
%
\documentclass[12pt,twoside]{article}

\renewcommand{\familydefault}{\sfdefault}
\usepackage[sfdefault]{FiraSans}
\usepackage{fontawesome}

\usepackage{graphicx}
\usepackage{amsmath}
\usepackage{dsfont}
\usepackage{pbox}

\input{macros}

\usepackage{color, colortbl, soul}
\usepackage{longtable}
\definecolor{Gray}{gray}{0.8}

\definecolor{Navy}{rgb}{0.0,0.1372549,0.24313725}
\definecolor{Gold}{rgb}{0.79607843,0.67843137,0.20784314}
\definecolor{Velvet}{rgb}{0.45882353,0.04313725,0.21176471}

\setlength{\oddsidemargin}{0pt}
\setlength{\evensidemargin}{0pt}
\setlength{\textwidth}{6.5in}
\setlength{\topmargin}{0in}
\setlength{\textheight}{8.5in}

% Fill these in!
\newcommand{\theproblemsetnum}{1}
\newcommand{\releasedate}{April 07, 2017~~}
\newcommand{\partaduedate}{April 07}
\newcommand{\tabUnit}{3ex}
\newcommand{\tabT}{\hspace*{\tabUnit}}

\begin{document}

\handout{\textsc{Monthly Report - March 2017}}{\releasedate}

\newif\ifsolution
\solutiontrue
\newcommand{\solution}{\textbf{My plan:}}

\begin{enumerate}
  \setlength{\itemsep}{0.1em}
  \item Research activities in March 2017.
  \item Research plan in April 2017.
\end{enumerate}

\setlength{\parindent}{0pt}

\medskip

\hrulefill

\textbf{Collaborators:}
%%% COLLABORATORS START %%%
Arie-san, Choong-san, Kaushalya-san.
%%% COLLABORATORS END %%%

\vspace{2em}

\section{Research activities in March}

\setulcolor{Gold}
\setul{0.5ex}{0.3ex}
\subsection*{\ul{Motif in Complex Network}}
\vspace{-1.5em}

\begin{center}
  \renewcommand{\arraystretch}{1.5}
  \begin{longtable}{p{0.8\textwidth} p{0.2\textwidth}}

  \textbf{Update current results.} I compiled some network datasets and run experiments
  on node labeling. These networks are quite large (YouTube, Wikipedia, Synthetic network)
  so the full motif analysis is not possible. My experiment is conducted assuming these
  networks have motifs similar to other (smaller) social networks and citation networks.
  & \pbox{0.2\textwidth}{\vspace{1em} {\color{Gold}March 01} \\ \faArrowRight \ {\color{Gold}Mar 16}} \\

  \textbf{Rebuttal IJCAI'17.} I have received and wrote 4 rebuttals to the reviewers.
  Although the discussion period has started for IJCAI'17 (after the reviewer received
  the rebuttals), I haven't received any reply from the reviewers as of the time this
  report is written. The rebuttals can be found here: {\color{Velvet}\texttt{https://goo.gl/pBj5dz}}.
  & \pbox{0.2\textwidth}{\vspace{1em} {\color{Gold}March 29} \\ \faArrowRight \ {\color{Gold}Mar 31}} \\

  \textbf{Wavelet basis.} To earn a deeper understanding of my methods and graph convolution
  method, I have reviewed my linear algebra. During this period, I studied discrete Fourier
  transformation, discrete Cosine transformation, wavelet transformation, and other related
  techniques. The highlight of my study here is that my method of using the motif Laplacian
  matrix as the basis for graph convolution can be understand as the dilation operation
  in convolutional neural network.
  & \pbox{0.2\textwidth}{\vspace{1em} {\color{Gold}March 10} \\ \faArrowRight \ {\color{Gold}April 03}} \\

  \end{longtable}
\end{center}

\pagebreak

\setulcolor{Navy}
\setul{0.5ex}{0.3ex}
\subsection*{\ul{CREST-Deep Compression}}
\vspace{-1.5em}

\begin{center}
  \renewcommand{\arraystretch}{1.5}
  \begin{longtable}{p{0.8\textwidth} p{0.2\textwidth}}

  \textbf{Study loss surface of a deep model.} Since my network compression approach is discrete
  optimization, I need to understand about: 1. Loss surface properties with respect to the network
  weights; and 2. The theoretical background to measure accuracies of a deep network. There are some
  interesting conclusion and insight from papers, which I will present in my future progress presentation.
  & \pbox{0.2\textwidth}{\vspace{1em} {\color{Navy}March 12} \\ \faArrowRight \ {\color{Navy}Mar 31}} \\

  \textbf{Build Caffe from source.} I have successfully build the deep learning tool Caffe from source
  and written a tutorial for building it on our website {\color{Velvet}\texttt{net-titech}}.
  & \pbox{0.2\textwidth}{\vspace{1em} {\color{Navy}March 25} \\ \faArrowRight \ {\color{Navy}Mar 26}} \\

  \textbf{Research Progress Presentation.} I have summarized the research progress and idea of
  our project members and presented in the CREST-Deep project meeting. The slide can be found
  here: \texttt{\color{Velvet}https://goo.gl/mKb7a7}.
  & \pbox{0.2\textwidth}{\vspace{1em} {\color{Navy}March 10} \\ \faArrowRight \ {\color{Navy}April 03}} \\

  \end{longtable}
\end{center}

\section{Research plan in April}

\setulcolor{Gold}
\setul{0.5ex}{0.3ex}
\subsection*{\ul{Motif in Complex Network}}
\vspace{-1.5em}

\begin{center}
  \renewcommand{\arraystretch}{1.5}
  \begin{longtable}{p{0.8\textwidth} p{0.2\textwidth}}

  \textbf{Fix paper according to reviewers.} As the reviewers of IJCAI suggested, I will
  make same changes in my paper. I will make one short and one longer version for possible
  journal submission.
  & \pbox{0.2\textwidth}{\vspace{1em} {\color{Gold}April 14} \\ \faArrowRight \ {\color{Gold}Apr 20}} \\

  \textbf{Determinantal Point Process.} Determinantal Point Process is a time-consuming process
  to select a diverse set of items (with respect to some pre-defined diversity scoring function).
  The application to graph of this process is still unexplored and vague (there are some work but
  the quality is not high). I will work with Arie-san on applying the DPP to complex network. My
  goal is to compare the DPP (time-consuming, good performance) with my diversity submodularity
  model (fast estimation, the performance might be lower than DPP). The result from here will give
  me great insight about network coverage, network summarization, and also experimental data for my thesis.
  & \pbox{0.2\textwidth}{\vspace{1em} {\color{Gold}April 20} \\ \faArrowRight \ {\color{Gold}Apr 30}} \\

  \end{longtable}
\end{center}

\setulcolor{Navy}
\setul{0.5ex}{0.3ex}
\subsection*{\ul{CREST-Deep Compression}}
\vspace{-1.5em}

\begin{center}
  \renewcommand{\arraystretch}{1.5}
  \begin{longtable}{p{0.8\textwidth} p{0.2\textwidth}}

  \textbf{Compress weight matrices using JPEG algorithms.} Following the professors' suggestion in my
  presentation on April 4th, I will recreate the result from deep neural network compression papers
  (Hans, Ullrich, etc.) and deploy cuda-JPEG to extract model on-demand. There can be some performance
  trade-off, but overall the result might be interesting as we will run compressed model on general
  purpose processors (Nvidia GPUs).
  & \pbox{0.2\textwidth}{\vspace{1em} {\color{Navy}April 05} \\ \faArrowRight \ {\color{Navy}Apr 21}} \\

  \textbf{Develop method to perform convolution with JPEG images.} The convolution operation can be
  done in the frequencies domain (i.e. using DCT coefficients) in the same manner as in the spatial
  domain. Taking advantage of this fact, we can directly use the compressed image or video as input
  and run on compressed network. This idea heavily depends on implementation, therefore I will contact
  Shinoda Lab to discuss about their video and image input format.
  & \pbox{0.2\textwidth}{\vspace{1em} {\color{Navy}April 22} \\ \faArrowRight \ {\color{Navy}Apr 30}} \\

  \end{longtable}
\end{center}

\end{document}
