\documentclass[a4paper,12pt]{article}

\usepackage[utf8]{inputenc}
\usepackage{graphicx}
\usepackage{xcolor}

%\usepackage{tgheros}
%\usepackage[defaultmono]{droidmono}

\usepackage{amsmath,amssymb,amsthm,textcomp}
\usepackage{enumerate}
\usepackage{multicol}
\usepackage{tikz}
\usetikzlibrary{arrows,calc}
\tikzset{
 >=stealth',
 help lines/.style={dashed, thick},
 axis/.style={<->},
 important line/.style={thick},
 connection/.style={thick, dotted},
}

\usepackage{CJKutf8}

\usepackage{geometry}
\geometry{total={210mm,297mm},
left=25mm,right=25mm,%
bindingoffset=0mm, top=20mm,bottom=20mm}


\linespread{1.3}

\newcommand{\linia}{\rule{\linewidth}{0.5pt}}

% custom theorems if needed
% my own titles
\makeatletter
\renewcommand{\maketitle} {
\begin{center}
\vspace{2ex}
{\LARGE \textsc{\@title}}
\vspace{1ex}
\\
\linia\\
\@author \hfill \@date
\vspace{4ex}
\end{center}
}
\makeatother
%%%

% custom footers and headers
\usepackage{fancyhdr}
\pagestyle{fancy}
\lhead{}
\chead{}
\rhead{}
\lfoot{ADE:Assignment 7}
\cfoot{Page \thepage}
\rfoot{15M54097}
\renewcommand{\headrulewidth}{0pt}
\renewcommand{\footrulewidth}{0pt}
%


% code listing settings
\usepackage{listings}
\lstset{
    language=SQL,
    basicstyle=\ttfamily\small,
    aboveskip={1.0\baselineskip},
    belowskip={1.0\baselineskip},
    columns=fixed,
    extendedchars=true,
    breaklines=true,
    tabsize=4,
    prebreak=\raisebox{0ex}[0ex][0ex]{\ensuremath{\hookleftarrow}},
    frame=lines,
    showtabs=false,
    showspaces=false,
    showstringspaces=false,
    keywordstyle=\color[rgb]{0.627,0.126,0.941},
    commentstyle=\color[rgb]{0.133,0.545,0.133},
    stringstyle=\color[rgb]{01,0,0},
    numbers=left,
    numberstyle=\small,
    stepnumber=1,
    numbersep=10pt,
    captionpos=t,
    escapeinside={\%*}{*)}
}


%\renewcommand*\thelstnumber{\arabic{lstnumber}:}

%\DeclareCaptionFormat{mylst}{\hrule#1#2#3}
%\captionsetup[lstlisting]{format=mylst, labelfont=bf, singlelinecheck=off, labelsep=space}

\begin{document}

\title{Advanced Data Engineering: Assignment 7}

\author{NGUYEN T. Hoang - SID: 15M54097}

\date{Fall 2015, W831 Tue. Period 5-6 \\ \hfill Due date: 2015/12/08}

\maketitle
\vspace{-4.5em}
\hspace{5.3em} (\begin{CJK}{UTF8}{min}ホアン\end{CJK})
\vfill
\section*{Problem}
\begin{itemize}
	\item Consider conditions for making disk-I/O costs of the GRACE hash join and Hybrid hash join superior to those of simple hash join, respectively (GRACE vs. Simple, Hybrid vs. Simple).	
	\item QUIZ: Consider an SQL sentence to derive the result given in the lecture note.
\end{itemize}


\vspace{3em}
\section*{Answer}
\noindent
Based on the cost estimation model and the \emph{Comparision of Hash Join Algorithms} table, we have the following notations:
\begin{equation}
	\mbox{Let: } \alpha = \frac{|M|}{|R|} \mbox{. I will make conditions based on } \alpha.
\end{equation}
The cost estimation of Simple Hash Join is re-written using $\alpha$ as follow:
\begin{itemize}
	\item Number of disk I/O: $$SH_{I/Os} = \frac{|R|}{|M|} \times (|R| + |S|) = \frac{1}{\alpha} \times (|R| + |S|)$$
	\item The cost of hashing:  $$SH_{Hash\_Cost} = \frac{|R| + |M|}{2|M|}(\{R\} + \{S\}) = \frac{\alpha + 1}{2\alpha} (\{R\} + \{S\}) $$
\end{itemize}
\vfill
\pagebreak




\section{GRACE Hash Join}

The cost estimation of GRACE Hash Join given in the lecture note is:
\begin{itemize}
	\item Number of disk I/O required: $$GH_{I/Os} = 3 \times (|R| + |S|)$$
	\item The cost of hashing: $$GH_{Hash\_Cost} = 2 \times (\{R\} + \{S\})$$
\end{itemize}

\noindent
\textbf{Disk I/O ratio} between GRACE Hash Join and Simple Hash Join is: $$ \mu_{GRACE-Simple} = \frac{GH_{I/Os}}{SH_{I/Os}} = 3 \times \alpha $$
\noindent
As we can see here, the disk I/O ratio between GRACE Hash Join and Simple Hash Join is directly proportional to our parameter $\alpha$. Intuitively this means that if we have a small memory and a large relation table (small $\alpha$), then the number of disk I/O performs by GRACE Hash Join will be much smaller than that of Simple Hash Join. We have:
\begin{itemize}
	\item $\alpha \leq \frac{1}{3}$ : GRACE Hash Join has equal or worse disk I/O cost than Simple Hash Join (larger number of Disk I/O). In this case, at least a third of the relational table can fit into memory.
	\item $\alpha < \frac{1}{3}$ : GRACE Hash Join has better disk I/O cost than Simple Hash Join (smaller number of Disk I/O). In this case, the size of the relational table is at least 3 times bigger than the memory.
	\item $\alpha \ll \frac{1}{3}$ : GRACE Hash Join is superior in term of I/O cost compare to Simple Hash Join. This is the practical case, where we have limitted amount of memory, but the size of the relational table is very big.
\end{itemize}




\section{Hybrid Hash Join}

The cost estimation of Hybrid Hash Join given in the lecture note is rewritten as follow:
\begin{itemize}
	\item Number of disk I/O required: $$HH_{I/Os} = (3-2\frac{|M|}{|R|}) \times (|R| + |S|) = (3-2\alpha) \times (|R| + |S|) $$
	\item The cost of hashing: $$HH_{Hash\_Cost} = 2 \times (\{R\} + \{S\})$$
\end{itemize}

\noindent
\textbf{Disk I/O ratio} between Hybrid Hash Join and Simple Hash Join is: $$ \mu_{Hybrid-Simple} = \frac{HH_{I/Os}}{SH_{I/Os}} = 3\alpha - 2\alpha^{2} $$
\noindent
This model is designed for practical settings, therefore we assume that $0 < \alpha < \frac{3}{2}$. The Disk I/O cost estimation of Hybrid Hash Join is stated as follow: 
\begin{itemize}
	\item $0 < \alpha < \frac{1}{2}$ : Hybrid Hash Join has better disk I/O cost than Simple Hash Join (smaller number of Disk I/O). 
	\item $\frac{1}{2} \leq \alpha \leq 1$ : Hybrid Hash Join has equal or worse disk I/O cost than Simple Hash Join. The worst case is when $\alpha = 3/4$.
	\item $1 < \alpha < \frac{3}{2}$ : Hybrid Hash Join has better disk I/O cost than Simple Hash Join.
\end{itemize}


\section{QUIZ: SQL sentence}

The SQL sentence to derive each saleman in R and his/her sale count of some product ID in the product table S is:

\begin{lstlisting}[label={list:first},SQL sentence to derive Saleman and number of sale]
SELECT Salesman, COUNT(*) FROM R 
WHERE ProductID IN (
	SELECT ProductID FROM S)
GROUP BY Salesman;
\end{lstlisting}


\end{document}
