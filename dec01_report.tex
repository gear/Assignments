%
% Montly report: December 2016
% Author: Hoang Nguyen
%
\documentclass[12pt,twoside]{article}

\usepackage{graphicx}
\usepackage{amsmath}
\usepackage{dsfont}

\input{macros}

\usepackage{color, colortbl}
\usepackage{longtable}
\definecolor{Gray}{gray}{0.8}

\setlength{\oddsidemargin}{0pt}
\setlength{\evensidemargin}{0pt}
\setlength{\textwidth}{6.5in}
\setlength{\topmargin}{0in}
\setlength{\textheight}{8.5in}

% Fill these in!
\newcommand{\theproblemsetnum}{1}
\newcommand{\releasedate}{December 01, 2016~~}
\newcommand{\partaduedate}{November 28}
\newcommand{\tabUnit}{3ex}
\newcommand{\tabT}{\hspace*{\tabUnit}}

\begin{document}

\handout{\textsc{Monthly Report - Dec 2016}}{\releasedate}

\newif\ifsolution
\solutiontrue
\newcommand{\solution}{\textbf{My plan:}}

\begin{enumerate}
  \item Research activities in November.
  \item Research plan in October.
\end{enumerate}

\setlength{\parindent}{0pt}

\medskip

\hrulefill

\textbf{Collaborators:}
%%% COLLABORATORS START %%%
None.
%%% COLLABORATORS END %%%

\vspace{1em}

\section{Research activities in November}

\begin{center}
  \renewcommand{\arraystretch}{1.5}
  \begin{longtable}{| c | p{6.5cm} | p{6.5cm} |}
  \hline
  & \textbf{\textsc{Motif-Aware Graph Embedding}} & \textbf{\textsc{Distributed Embedding}} \\ \hline
  \textbf{\textsc{Topic}} & Improve graph embeddings by using dominant motifs. & 
  Collaboration with Defago-sensei's lab on distributed computing.\\ \hline
  \textbf{\textsc{Idea}} & Comparing my embedding method with other existing methods
  on new planar graph datasets and bipartite datasets.
  & Application of graph embedding on a distributed environment and computing
  the embedding in a distributed settings. \\ \hline
  \textbf{\textsc{Activities}} & Firstly, I have published my code on github \footnote{\texttt{https://github.com/gear/motifwalk/}}.
  Secondly, I have revised my previous implementations and test procedure. I also have
  revised other authors' implementations. Finally, I have chosen new undirected graph datasets
  with community ground truth to compare my algorithm with. I also used the artificial
  network generated from Choong-san's master thesis.
  & I have discussed with Professor Defago about some possible collaboration
  between network/machine learning and distributed algorithm. Currently I am reading
  Professor Defago's provided materials about distributed graph algorithms. One of
  the interesting application is to propose a graph embedding algorithm in a distributed
  environment. Another is to study the bound for Byzantine General Problem on an arbitary
  network topology.\\ \hline
  \end{longtable}
\end{center}

\section{Research plan in December}

\begin{center}
  \renewcommand{\arraystretch}{1.5}
  \begin{longtable}{| c | p{12cm} |}
  \hline
  & \textbf{\textsc{December - 2016}} \\ \hline
  \textbf{\textsc{Topic}} & Random processes on graph and distributed graph algorithms.\\ \hline
  \textbf{\textsc{Plan}} & I will continue to study graph embeddings and the underlying
  randomness in this algorithm mathematically. Besides, I will also consider a distributed
  solution for the graph embedding problem, possibly application using GPU.
  \\ \hline
  \end{longtable}
\end{center}

\end{document}
