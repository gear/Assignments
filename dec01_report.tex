%
% Montly report: December 2016
% Author: Hoang Nguyen
%
\documentclass[12pt,twoside]{article}

\usepackage{graphicx}
\usepackage{amsmath}
\usepackage{dsfont}

\input{macros}

\usepackage{color, colortbl}
\usepackage{longtable}
\definecolor{Gray}{gray}{0.8}

\setlength{\oddsidemargin}{0pt}
\setlength{\evensidemargin}{0pt}
\setlength{\textwidth}{6.5in}
\setlength{\topmargin}{0in}
\setlength{\textheight}{8.5in}

% Fill these in!
\newcommand{\theproblemsetnum}{1}
\newcommand{\releasedate}{December 01, 2016~~}
\newcommand{\partaduedate}{November 28}
\newcommand{\tabUnit}{3ex}
\newcommand{\tabT}{\hspace*{\tabUnit}}

\begin{document}

\handout{\textsc{Monthly Report - Dec 2016}}{\releasedate}

\newif\ifsolution
\solutiontrue
\newcommand{\solution}{\textbf{My plan:}}

\begin{enumerate}
  \item Research activities in November.
  \item Research plan in October.
\end{enumerate}

\setlength{\parindent}{0pt}

\medskip

\hrulefill

\textbf{Collaborators:}
%%% COLLABORATORS START %%%
None.
%%% COLLABORATORS END %%%

\vspace{1em}

\section{Research activities in November}

\begin{center}
  \renewcommand{\arraystretch}{1.5}
  \begin{longtable}{| c | p{6.5cm} | p{6.5cm} |}
  \hline
  & \textbf{\textsc{Motif-Aware Graph Embedding}} & \textbf{\textsc{Distributed Embedding}} \\ \hline
  \textbf{\textsc{Topic}} & Improve graph embeddings by using dominant motifs. & 
  Collaboration with Defago-sensei's lab on distributed computing.\\ \hline
  \textbf{\textsc{Idea}} & Comparing my embedding method with other existing methods
  on new planar graph datasets and bipartite datasets.
  & Application of graph embedding on a distributed environment and computing
  the embedding in a distributed settings. 
  \textbf{\textsc{Activities}} & Firstly, I have published my code \texttt{[https://github.com/gear/motifwalk]}.
  Secondly, I have revised my previous implementations and test procedure. I also have
  revised other authors' implementations. Finally, I have chosen new undirected graph datasets
  with community ground truth to compare my algorithm with. I decided to use the 

  & I have assembled a team
  of 4 to participate
  in WSDM Cup 2017. This year WSDM Cup 2017 consists of 2 task:
  Vandalism detection on Wikidata and Triple scoring. Details
  are given in the reference.\\ \hline
  \end{longtable}
\end{center}

\section{Research plan in November}

\begin{center}
  \renewcommand{\arraystretch}{1.5}
  \begin{longtable}{| c | p{12cm} |}
  \hline
  & \textbf{\textsc{November - 2016}} \\ \hline
  \textbf{\textsc{Topic}} & Rare event detection, knowledge graph,
  and submodular models on graph.\\ \hline
  \textbf{\textsc{Plan}} & About the WSDM Cup 2017, we will finalize our approach
  and create a prototype for submission by the end of November. About my research,
  by the end of November, I will have substantial knowledge about random processes
  on graph and write a literature review.
  \\ \hline
  \end{longtable}
\end{center}

\end{document}
